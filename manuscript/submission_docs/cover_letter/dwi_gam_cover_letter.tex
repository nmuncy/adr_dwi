\documentclass{article}

\usepackage{lipsum}
\providecommand\UNLtonamefirst{Stephen}
\providecommand\UNLtonamelast{Smith}
\providecommand\UNLtonametitle{Dr.}
\providecommand\UNLtonamedegree{Ph.D.}
% \providecommand\UNLtoname{Dr. Stephen Smith}
\providecommand\UNLtoaddress{
    Editor-in-Chief\\
    \textit{Imaging Neuroscience}
}
\usepackage[]{xcolor}
\usepackage[top=2in,left=1.5in,bottom=0.5in,right=0.625in]{geometry}
\usepackage{graphicx}
\graphicspath{ {./}}
\usepackage[colorlinks=false,pdfborder={0 0 0},]{hyperref}
\usepackage[absolute]{textpos}
\usepackage{ifthen}
\usepackage{soul}
\usepackage{tikz}
\usetikzlibrary{calc}

% Remove paragraph indentation
\parindent0pt
\setlength{\parskip}{0.8\baselineskip}
\raggedright
\pagestyle{empty}
% Ensure consistency in the footer
\urlstyle{sf}

% From info
\providecommand\UNLfromname{Nathan M. Muncy}
\providecommand\UNLfromtitle{Research Assistant Professor}
\providecommand\UNLfromdegree{Ph.D.}
\providecommand\UNLfromdept{Center for Brain, Biology and Behavior}
\providecommand\UNLfromaddress{B78 East Stadium, P.O. 880156, Lincoln, NE 68588-0156}
\providecommand\UNLfromtel{(402) 472-0198}
% \providecommand\UNLfromfax{}
\providecommand\UNLfromemail{\url{nmuncy2@unl.edu}}
\providecommand\UNLfromweb{\url{https://cb3.unl.edu}}

% Formalities
\providecommand\UNLdate{\today}
\providecommand\UNLopening{Dear \UNLtonametitle\ \UNLtonamelast,}
\providecommand\UNLclosing{Sincerely,}
% Update this and the next line to the correct path
% \providecommand\UNLsignaturefile{AleeSignatureVector}
\providecommand\UNLsignaturefile{}

% Branding
\providecommand\UNLlogofile{R-UN_L_4-c.jpg}
\definecolor{UNLscarlet}{RGB}{208,0,0} % https://ucomm.unl.edu/brand/templates-assets/colors/
\definecolor{UNLgray}{RGB}{199,200,202}
% \providecommand\UNLenclosure{}
\usepackage{Oswald}
\linespread{1.05}

\usepackage{fancyhdr}
\pagestyle{fancy}

\renewcommand{\footrulewidth}{0.7pt}
\renewcommand{\footrule}{\hbox to \headwidth{\color{UNLscarlet}\leaders\hrule height \footrulewidth\hfill}}
\fancyfoot{}
\fancyfoot[C]{%
    {\footnotesize\color{UNLgray}\sffamily
    \UNLfromaddress\ \textbullet\ %\\[-0.1\baselineskip]
    \UNLfromtel\ \textbullet\ \UNLfromweb\ \textbullet\ \UNLfromemail}\color{black}}

\fancyhead{}
\fancyhead[L]{%
    \begin{textblock*}{2in}[0.3066,0.39](1.5in,1.33in)
        \includegraphics[width=2in]{\UNLlogofile}
    \end{textblock*}
    \begin{textblock*}{6.375in}(1.5in,1.65in)
        \sffamily
        \hfill \color{UNLgray} \UNLfromdept
    \end{textblock*}
    \begin{tikzpicture}[remember picture,overlay]
        \draw[color=UNLscarlet,line width=0.7pt] (current page.north west)+(2.95in,-1.6in) -- ($(-0.625in,-1.6in)+(current page.north east)$);
    \end{tikzpicture}}
\renewcommand{\headrulewidth}{0pt}


\AtBeginDocument{
    % Text lines should be less than 6in long
    \newgeometry{top=2in,left=1.5in,bottom=1.2in,right=1in}

    % \UNLdate\\
    % \bigskip
    \UNLtonamefirst\ \UNLtonamelast, \UNLtonamedegree \hfill \UNLdate\\
    % \UNLtoname\ifthenelse{\equal{\UNLtoname}{}}{}{\\}
    \UNLtoaddress
    \bigskip

    \UNLopening \par
    }

% \AtEndDocument{
%     \par\vspace{2ex}
%     \UNLclosing

%     \ifthenelse{
%         \equal{\UNLsignaturefile}{}}
%         {\bigskip\bigskip}{\includegraphics[width=1.2in]{\UNLsignaturefile}\\[-0.2\baselineskip]}

%     \UNLfromname, \UNLfromdegree \\
%     \UNLfromtitle\ifthenelse{\equal{\UNLfromtitle}{}}{}{\\}
%     % \UNLenclosure
%     }

\begin{document}
Please find attached our manuscript entitled ``Modeling Sport-Related Concussion Using Hierarchical Generalized Additive Models: A Longitudinal Diffusion MRI Study in College Athletes''. Diffusion-weighted imaging is a common technique employed to study concussion-related axonal changes, but the standard statistical methods lack sensitivity to model the dimensionality, interdependence, and spatial dynamics of such sequelae. The aim of this manuscript is to demonstrate that Hierarchical Generalized Additive Models (HGAMs) resolve such issues and promote multimodal investigations, yielding novel insights into the nature of concussion injury and recovery. This extends our previous work with Generalized Additive Models and is the first time HGAMs have been employed both diffusion MRI or concussion-related data.

To illustrate the utility and sensitivity of HGAMs, we utilized a set of MRI and clinical assessment data collected from 69 collegiate athletes at three time points: baseline, post-concussion, and return-to-play. Probabilistic tractography was conducted via Automated Fiber Quantification to aid within-tract analyses, and three levels of HGAMs tested for concussion-related changes. First, a whole-brain longitudinal model allowed for pooling variance within subjects and across time to detect injury- and recovery-related microstructural changes to fractional anisotropic (FA) values. Second, tracts with injury-related FA changes were modeled with tract- and scalar-specific longitudinal models to elucidate the source and nature of FA changes. Finally, multimodal models utilizing tensor product interaction smooths related changes in tract scalars to clinical assessment metrics and recovery time.

Our analyses reveal spatially nonlinear changes in tract scalar profiles that relate to both injury and recovery, importantly elucidating the time course of concussion sequelae, their relationship to clinical assessment metrics, and the nature of microstructural disruption. Such findings would not have been possible with traditional techniques, which showcases the relevance of HGAMs to sensitive analyses involving diffusion MRI. This manuscript would be of broad interest to the \textit{Imaging Neuroscience} community as the statistical methods employed resolve statistical issues that occur when modeling complex interactions across multiple factors and are applicable to any dataset where nonlinear or complex dynamics are suspected.

\par\vspace{2ex}
Sincerely,

\bigskip\bigskip

Nathan M. Muncy\\
Heather C. Bouchard\\
Douglas H. Schultz\\
Maital Neta\\
Aron K. Barbey


\end{document}