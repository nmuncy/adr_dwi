% doc type
\documentclass[12pt]{article}

% set up supplemental section
\newcommand{\beginsupplement}{%
	\setcounter{table}{0}
	\renewcommand{\thetable}{S\arabic{table}}%
	\setcounter{figure}{0}
	\renewcommand{\thefigure}{S\arabic{figure}}%
}

% format page, paragraph
\usepackage[margin=1in]{geometry}
\usepackage{lineno}
\linenumbers
\usepackage{setspace}
\doublespacing
\usepackage[utf8]{inputenc}
\usepackage[hidelinks]{hyperref}
\usepackage{authblk}
% \usepackage[most]{tcolorbox}
% \usepackage{cleveref}

% \tcbset{theostyle/.style={
%     enhanced,
%     sharp corners,
%     attach boxed title to top left={xshift=-1mm, yshift=-4mm, yshifttext=-1mm},
%     top=1.5ex,
%     colback=lightgray,
%     colframe=black,
%     fonttitle=\bfseries,
%     boxed title style={sharp corners,size=small,colback=blue!75!black,colframe=blue!75!black}
% }}

% \newtcbtheorem[]{Definition}{Box}{%
%   theostyle
% }{def}

% bib - uses better biblatex
\usepackage[style=apa,backend=biber]{biblatex}
\addbibresource{./Zotero_adr_dwi.bib}

% figures and tables
\usepackage{graphicx}
\graphicspath{ {./} }
\usepackage{multirow}
\usepackage[table,xcdraw]{xcolor}
\usepackage{url}
\usepackage{float}
\usepackage{longtable}
\usepackage{rotating}

% code
\usepackage{listings,lstautogobble}
\lstset{language=R,
	basicstyle=\small\ttfamily,
	otherkeywords={0,1,2,3,4,5,6,7,8,9},
	morekeywords={TRUE,FALSE},
	deletekeywords={data,frame,length,as,character},
	autogobble=true
}
\renewcommand\lstlistingname{R Code}

% math
\usepackage{amsmath}
\usepackage{caption}
\DeclareCaptionType{equ}[R Code][]

% title page info
\title{A Novel Method for Examining Longitudinal Concussion-related Diffusion MRI Changes in Collegiate Athletes: Evidence for Stability, Worsening, and Recovery of Different Tracts}
\date{}

\author[1,2,*]{Nathan M. Muncy}
\author[1,2]{Heather C. Bouchard}
\author[1,2]{Doug H. Schultz}
\author[1,2]{Maital Neta}
\author[1,2]{Aron K. Barbey}

\affil[1]{Center for Brain, Behavior and Biology, University of Nebraska-Lincoln}
\affil[2]{Department of Psychology, University of Nebraska-Lincoln}
\affil[*]{Corresponding author.	Email: nmuncy2@unl.edu}

% start document
\begin{document}

% title page
\maketitle
\pagebreak


% abstract page
\begin{abstract}
Sports-related traumatic brain injuries affect approximately 3.8 million individuals in the US each year, each triggering a complex sequence of axonal changes. Although diffusion-weighted imaging can quantify these changes, these longitudinal data are difficult to model statistically given issues of dimensionality, interdependence, and non-linear interactions. Proposal: Hierarchical generalized additive models (HGAMs) are well-suited to fit such data with the requisite flexibility and sensitivity to investigate (a) the spatial and temporal changes of white matter tracts, and (b) how such changes relate to diagnostic assessments. Methods: We used MRI and ImPACT data collected from college athletes at three visits: start-of-season, post-concussion, and return-to-play. Diffusion tensors were modeled via constrained spherical deconvolution and probabilistic tractography from PyAFQ yielded 100 scalar values per white matter bundle. Results: By fitting the scalar profiles with longitudinal HGAMs we detected within-tract changes as a function of visit, revealing distinct patterns of post-injury disruption and recovery. Further, we examined whether these evolving diffusion metrics correlated with cognitive outcomes and found moderate evidence linking white matter alterations to ImPACT composite scores. Merit: HGAMs offer a powerful framework to capture the complex progression of brain injury. Our findings suggest that HGAMs enhance our understanding of the spatiotemporal dynamics of brain injury and may enable more accurate tracking of injury and recovery.

\end{abstract}

\vfill
KEYWORDS: DWI, MRI, GAM, TBI, Concussion\\
\pagebreak


\section{Introduction}
\label{sec:intro}

% Epidemilogic description of TBI in US, athletes
Each year an estimated 1.6 to 3.8 million sports-related concussions occur in the United States \parencite{langlois2006EpidemiologyImpactTraumatic,daneshvar2011EpidemiologySportRelatedConcussion,barkhoudarian2016MolecularPathophysiologyConcussive}. In collegiate sport, concussions during competition occur at a rate of 30.09 per 10,000 athletic events in men's American football and 14.23 per 10,000 athletic events in women's soccer \parencite{pierpoint2021EpidemiologySportRelatedConcussion}. While many concussions (e.g. mild traumatic brain injury (mTBI); see \textcite{mayer2017SpectrumMildTraumatic} and \textcite{silverberg2023AmericanCongressRehabilitation}) may still go under-reported, the incidence of sport-related concussions has risen in recent years \parencite{coronado2015TrendsSportsRecreationRelated,pierpoint2021EpidemiologySportRelatedConcussion}, likely due to increased awareness, evolving diagnostic criteria, improved symptom recognition, and shifts in athletic participation \parencite{yang2017NewRecurrentConcussions}. Despite this growing attention, the understanding of the brain's physiological response through traditional clinical concussion assessments remains limited. Diagnosis of concussions typically involve a multifaceted process to collect different types of information including patient history, neurological examination, symptom inventories, and cognitive testing \parencite{patricios2023ConsensusStatementConcussion}. However, these methods often fall short in detecting the subtle and complex pathophysiological changes that occur post-injury.

% mTBI sequelae description
Sport-related injury of the head can result from the rapid application of shear, tensile, and/or compressive strains that are consequential for sensitive neural structures, particularly neuronal axons that are especially vulnerable given their cytoskeletal structure \parencite{elsayed2008BiomechanicsTraumaticBrain,johnson2013AxonalPathologyTraumatic,bar-kochba2016StrainRatedependentNeuronal}. Collectively, the various biochemical cascades associated with axonal injury are referred to as diffuse or traumatic axonal injury \parencite{krieg2023IdentifyingPhenotypesDiffuse}, although `diffuse' is a misnomer as axonal injury is commonly found in parasagittal fibers e.g. splenium of corpus callosum \parencite{jang2020DiagnosticProblemsDiffuse,fork2005NeuropsychologicalSequelaeDiffuse,meythaler2001CurrentConceptsDiffuse,johnson2013AxonalPathologyTraumatic,lindsey2023DiffusionWeightedImagingMild}. Initially, disruption may be evident as axons are stretched or sheared beyond their compensatory mechanisms \parencite[e.g. spectrin elongation;][]{dubey2020AxonalActinspectrinLattice}, resulting in primary injuries of axolemmal mechanoporation, disconnection, and disruption as well as breakage of actin-spectrin complexes, microtubules, and neurofilaments. Secondary injury mechanisms are then triggered by changes to the cytoskeletal and axolemmal properties. Microtubule disruption results in the accumulation of axonal transport products, and increased intracellular calcium activates proteolytic calpains which results in neurofilament compaction and spectrin degradation \parencite{shin2020AxonalTransportDysfunction,ma2013RoleCalpainsInjuryinduced}. Further, primary and secondary changes to ionic channels and exchangers can result in increased intracellular osmolarity, causing water to move from the extracellular to intracellular space, resulting in swelling and even cytotoxic edema \parencite{baracaldo-santamaria2022RevisitingExcitotoxicityTraumatic}. Separately, dissociation of the myelin sheathes may occur at nodal regions due to proteolysis of their binding proteins and sites \parencite{krieg2023IdentifyingPhenotypesDiffuse,song2022ConcussionLeadsWidespread}. While such primary and secondary sequelae are sufficient to trigger caspase-mediated axotomy and even apoptosis, this is not necessarily the case and neurons have the machinery to repair themselves \parencite{franze2013MechanicsNeuronalDevelopment,sakai2019InflammationNeuralRepair}. The in vivo investigation of such sequelae in human patients, however, is not directly accessible but of note is that these various injury mechanisms result in differential diffusion trajectories for water molecules. For instance, cellular edema would result in increased tortuosity as axoplasmic water increases and free extracellular decreases, while axotomy would result in increased extracellular water \parencite{rosenblum2007CytotoxicEdemaMonitoring,liang2007CytotoxicEdemaMechanisms,borja2018DiffusionMRImaging,barkhoudarian2016MolecularPathophysiologyConcussive}.

% DWI and mTBI
Diffusion tensor imaging (DTI) is a widely used technique to study white matter fibers. By measuring water diffusion trajectories in para-orthogonal directions, diffusion tensors can be constructed and combined to construct models of tracts \parencite{danielian2010ReliabilityFiberTracking,reid2022TractspecificStatisticsBased,sarwar2019MappingConnectomesDiffusion,tournier2007RobustDeterminationFibre}. Axial diffusivity (AD) refers to the principal direction of diffusion ($\lambda_1$), while the other two transverse directions ($\lambda_2$, $\lambda_3$) are averaged to calculate radial diffusivity (RD). Likewise, mean diffusivity (MD) is the average of $\lambda_{1-3}$, and fractional anisotropy (FA) is calculated from $\lambda_{1-3}$ to describe the total anisotropy of water diffusion. Within white matter, the diffusion trajectory of water is typically constrained by the axolemma such that $\lambda_1$ (AD) is parallel with the axon ($\lambda_\parallel$) and RD is perpendicular ($\lambda_\perp$) \parencite{mori1999ThreedimensionalTrackingAxonal,lilja2014VisualizingMeyersLoop,lindsey2023DiffusionWeightedImagingMild}. While orders of magnitude exist between the axonal diameter and voxel size, axonal fiber coherence results in tensors that can be used to algorithmically generate tract profiles which approximate the microstructural organization of white matter \parencite{kiselev2021MicrostructureDiffusionMRI,novikov2019QuantifyingBrainMicrostructure}. When mTBI disrupts axonal structure or axolemmal permeability, the resulting atypical diffusion can be measured with DTI and will be reflected in the respective tensors and corresponding tract profiles \parencite{macdonald2007DetectionTraumaticAxonal,macdonald2007DiffusionTensorImaging}.

% Modeling DWI, issues resolved by GAMs
Analysis of Fiber-Tract Quantification \parencite[PyAFQ;][]{yeatman2012TractProfilesWhite,kruper2021EvaluatingReliabilityHuman,kruper2024TractometryHumanConnectome} is a software package that generates white matter tract profiles from preprocessed DWI data. PyAFQ is of particular relevance for concussion research given its approach of subdividing white matter bundles into a number of equidistant nodes and then calculating scalar values for the local, within-tract environment. That is, as concussion-related axonal injury is typically found within a subregion of a given tract \parencite[e.g. parasagittal splenium;][]{jang2020DiagnosticProblemsDiffuse,johnson2013AxonalPathologyTraumatic,lindsey2023DiffusionWeightedImagingMild}, PyAFQ is capable of providing scalar values (e.g. FA) for the injured region that reflect concussion-related changes in diffusion. Such a within-tract node approach is in contrast to other popular methods of diffusion analyses \parencite{lindsey2023DiffusionWeightedImagingMild} which may not be sufficiently sensitive to detect within-tract scalar changes \parencite[e.g. whole-tract or skeletal analyses;][]{smith2006TractbasedSpatialStatistics,fischl2012FreeSurfer} or may not reflect the injury locale (region-of-interest mask). However, while the within-tract resolution afforded by PyAFQ is highly desirable for studying in vivo concussion-related axonal sequelae, a number of statistical considerations become relevant with PyAFQ-derived tractometric profiles: the shape of the profile (non-linear node-scalar interaction), interdependence of node scalars, non-Gaussian scalar distributions, and multiple comparisons with their corresponding corrections \parencite{muncy2022GeneralAdditiveModels,richie-halford2021MultidimensionalAnalysisDetection}.

Generalized additive models \parencite[GAMs;][]{wood2017GeneralizedAdditiveModels,pedersen2019HierarchicalGeneralizedAdditive,stasinopoulos2008GeneralizedAdditiveModels} are an extension of general linear models capable of modeling high-dimensional data which contain non-linear relationships. Where regression models fit data with a linear function, GAMs construct a smooth curve to fit data from a set of basis functions (i.e. splines; also see \textcite{verbyla1999AnalysisDesignedExperiments}). These smooths can capture complex X-Y relationships that would be underfit by models with linear assumptions. Further, non-linear three-way relationships can be modeled via surfaces termed a `tensor product interaction smooth', and hypersurfaces can model even higher dimensional interactions \parencite{baayen2020IntroductionGeneralizedAdditive}. Such capabilities have made GAMs useful in fields such as ecology, paleontology, and linguistics \parencite[e.g.][]{simpson2018ModellingPalaeoecologicalTime,pedersen2019HierarchicalGeneralizedAdditive,schmidt2011SpatiallyExplicitHeight,wieling2011QuantitativeSocialDialectology,simpson2018ModellingPalaeoecologicalTime,wieling2018AnalyzingDynamicPhonetic,murase2009ApplicationGeneralizedAdditive,vanrij2019AnalyzingTimeCourse}, which often model complex non-linear data in high dimensions and/or across multiple factors, and researchers using MRI techniques are beginning to adopt the statistical approach \parencite[e.g.][]{lee2025AtypicalMaturationFunctional,xu2025AgeBSASexspecific,wierenga2018UnravelingAgePuberty,mundo2022GeneralizedAdditiveModels,roy2025DevelopmentArcuateFasciculus,sorensen2021MetaanalysisGeneralizedAdditive,caffarra2024DevelopmentAlphaRhythm}. We previously demonstrated that GAMs are well-fit to model DTI tractometric profiles and provided details to guide decision points in model specification to aid their implementation \parencite{muncy2022GeneralAdditiveModels}. In that work, we used GAMs to investigate cross-sectional group tractometric differences, showcasing the sensitivity and utility of GAMs to detect subtle within-tract group differences in a principled fashion that accounted for the large number of tract nodes, the distribution of scalar values, and avoided the need for `point-wise' comparisons.

% Purpose - (a) extend GAMs for longitudinal DWI, (b) tensor product interaction smooths for multi-modal analyses
Here, we extend the use of GAMs to study concussion injury and recovery in a unique longitudinal dataset that consists of data from collegiate athletes at three visits: baseline (before an athlete sustained a concussion), post-concussion (within approximately 48 hours after diagnosed concussion), and return-to-play (after an athlete was medically cleared to return to full contact). First, we employed hierarchical GAMs to conduct a whole-brain, longitudinal analysis of concussion injury- and recovery-related diffusion (FA) changes. As disruption may occur across multiple tracts, pooling within-subject variance across tracts and visits is critical to better understand injury and recovery; modeling tracts individually would lose such variance, potentially inflating the Type-II error. Then, tracts which differed from baseline were further interrogated in post-hoc analyses to determine whether the change in FA was driven by AD ($\lambda_\parallel$) or RD ($\lambda_\perp$). Second, we used tensor product interaction smooths to build multimodal models with which we tested whether changes in FA related to clinical assessment metrics and recovery time. Such models helped elucidate the connection between physical injury and associated clinical assessments.

Our analyses were guided by mechanistic models of axonal injury and empirical findings from prior concussion research, which together suggest that white matter disruption is (a) spatially localized, (b) evolves nonlinearly over time, and (c) may recovery partially. We hypothesized that: (i) Concussion would lead to decreased FA in the posterior corpus callosum, which would show partial recovery by the return-to-play visit; (ii) FA decreases would be driven primarily by increased RD, rather than AD, consistent with myelin or membrane disruption rather than axotomy; and (iii) FA changes in affected regions would correlate with clinical outcomes, including cognitive performance and symptom burden, particularly during the acute phase (Figure \ref{fig:intro-hyp}) as well as recovery time. By combining high-resolution tractography with longitudinal modeling and theory-informed predictions, this study aims to characterize the spatial and temporal trajectory of white matter injury and its clinical relevance following sport-related concussion.

\begin{figure}[H]
	\centering
	\fbox{\includegraphics[width=0.95\textwidth]{fig_hypotheses.png}}
	\caption{Simulated data to relate hypotheses to PyAFQ-derived tract profiles and GAM smooths, tensor products. \textbf{Left:} simulated PyAFQ tract profiles of FA (top) and RD (bottom) at three visits: baseline (Red), post-concussion (Green), and return-to-play (Blue). Injury resulted in decreased FA values for nodes 60-80 at Post (top), and FA values are mostly (but not entirely) recovered to Base values by RTP. These FA changes relate to corresponding changes in RD (bottom). \textbf{Middle:} smooths produced by a hierarchical generalized additive model of the tract FA profiles across all visits. The overarching curvature of the tract profile is captured as a main effect in the global fit smooth (top). FA values between nodes 60-80 are lower in Post than Base as evidenced in the difference smooth (middle), and these values are partially recovered at RTP (bottom). \textbf{Right:} a simulated interaction between tract FA values and behavioral response modeled with a tensor product interaction smooth. A linear interaction between behavior and FA values is present in the region of nodes 60-80: higher behavioral values are associated with higher FA values (yellow) while lower behavioral values are related to lower FA values in the same region (blue). Third-order polynomial splines were used to simulate the tract profiles, and behavioral values were shifted by the smallest FA values among the injured nodes. Tract Node = subregion of a segmented white matter tract. FA = fractional anisotropy, RD = radial diffusivity. Base = baseline, Post = post-concussion, RTP = return-to-play.}
	\label{fig:intro-hyp}
\end{figure}


\section{Methods}
\label{sec:meth}

\subsection{Participants}
\label{ssec:meth-part}
Three hundred thirty-six National Collegiate Athletic Association (NCAA) athletes were recruited from men's football and women's soccer programs at the University of Nebraska-Lincoln. Of these participants, 69 (9 female, age = 19.36 $\pm$1.67, range = 17-24) experienced a sport-related concussion during the season and were included in analyses and results reported below. Additional demographic metrics (e.g. race, ethnicity, SES) are omitted to protect participant confidentiality as University athletes are public figures and identification may cause deleterious consequences (but a subset has been reported by \cite{bouchard2024ConcussionRelatedDisruptionsHub}). Due to the limited number of females, and the sport confound that all soccer athletes were female, we combined all participants into a single group. Institutional Review Board approval was obtained at the outset of the study, and prior to beginning experimental procedures participants completed informed consent and assent. MRI and clinical assessment (ImPACT) data were acquired during three visits: enrollment at the beginning of the season (baseline, Base), within approximately 48 hours of diagnosed concussion (post-concussion, Post), and prior to return-to-play (RTP). As MRI and ImPACT (below) data were gathered separately, a number of participants did not contribute MRI and/or ImPACT data across one or more of the visits. Total counts are provided in Table \ref{tbl:meth-demo}.

\begin{table}[H]
	\scriptsize
	% Please add the following required packages to your document preamble:
% \usepackage{multirow}
% \usepackage[table,xcdraw]{xcolor}
% Beamer presentation requires \usepackage{colortbl} instead of \usepackage[table,xcdraw]{xcolor}

\begin{tabular}{lccc}
Visit & \multicolumn{1}{l}{Sex} & \multicolumn{1}{l}{MRI} & \multicolumn{1}{l}{ImPACT} \\ \hline
 & M & 58 & 56 \\
\multirow{-2}{*}{Base} & \cellcolor[HTML]{C0C0C0}F & \cellcolor[HTML]{C0C0C0}9 & \cellcolor[HTML]{C0C0C0}5 \\
\rowcolor[HTML]{EFEFEF}
\cellcolor[HTML]{EFEFEF} & M & 57 & 45 \\
\rowcolor[HTML]{C0C0C0}
\multirow{-2}{*}{\cellcolor[HTML]{EFEFEF}Post} & {\color[HTML]{333333} F} & {\color[HTML]{333333} 8} & {\color[HTML]{333333} 3} \\
 & M & 49 & 30 \\
\multirow{-2}{*}{RTP} & \cellcolor[HTML]{C0C0C0}F & \cellcolor[HTML]{C0C0C0}7 & \cellcolor[HTML]{C0C0C0}3
\end{tabular}

	\caption{Number of athletes that contributed MRI and ImPACT data across all visits. Base = baseline, Post = post-concussion, RTP = return-to-play. M = Male, F = Female.}
	\label{tbl:meth-demo}
\end{table}


\subsection{ImPACT}
\label{ssec:meth-imp}
We assessed self-reported symptoms and cognitive performance using the Immediate Post-Concussion Assessment and Cognitive Testing (ImPACT), one of the most widely used tools for evaluating concussions \parencite{lovell2005ImPACT200540,dessy2017ReviewAssessmentScales}. Self-reported symptoms were measured with the 22-item Post-Concussion Symptom Scale within ImPACT \parencite{lovell2006MeasurementSymptomsFollowing}. Cognitive performance was assessed through five composite scores derived from ImPACT's computerized neurocognitive tests: verbal memory, visual memory, visual-motor processing speed, impulse control, and reaction time \parencite{lovell2005ImPACT200540}. These assessments were conducted in collaboration with clinicians from our Department of Athletic Medicine and were administered to participants at Base, Post, and throughout their recovery, with the participant's final assessment serving as RTP.


\subsection{MRI Protocol}
\label{ssec:meth-mri}
Magnetic Resonance Imaging data were collected on a 3-Tesla Siemens MAGNETOM Skyra scanner at the Center for Brain, Behavior and Biology (University of Nebraska-Lincoln) utilizing a 20-channel coil. For each of three visits (Base, Post, and RTP), participants contributed T1 and diffusion weighted images (T1w, DWI). T1w Multi-Echo Magnetization Prepared - RApid GRadient Echo (MEMP-RAGE) structural scans were acquired with the following parameters: TR = 2530 ms, TE = 1.69, 3.55, 5.41, and 7.27 ms, flip angle = 7$^{\circ}$, voxel size = 1 mm$^3$, FoV = 256 $\times$ 256, slices = 176 interleaved. DWI scans were acquired via TR = 3000 ms, TE = 95 ms, flip angle = 90$^{\circ}$, voxel size = 1.719 $\times$ 1.719 $\times$ 2.4 mm$^3$, 134 slices, multi-band acceleration factor = 3, directions = 128, bandwidth = 1500 Hz/Px, shells = 1 (b-value = 1000 s/mm$^2$), reference volumes = 6 (b-values = 0 s/mm$^2$; b$_0$). A set of field maps for the DWI scans were collected using the same acquisition direction (anterior-posterior; AP) and reversed (posterior-anterior; PA).


\subsection{MRI Data Processing}
\label{ssec:meth-mri-proc}
Preprocessing and modeling of the DWI data were conducted using FSL v6.0 \parencite{jenkinson2012Fsl} and PyAFQ v1.3.6 \parencite{kruper2021EvaluatingReliabilityHuman,yeatman2012TractProfilesWhite}. First, b$_0$ volumes and acquisition parameters were extracted and combined from the AP and PA field maps, and \lstinline{topup} used the resulting AP-PA b$_0$ file to calculate a distortion correction matrix. Next, a brain mask was constructed via \lstinline{bet}, and then preprocessing of the DWI data was conducted with \lstinline{eddy_openmp}, which incorporated the distortion correction matrix, brain mask, and a volume-acquisition parameter mapping index to produce motion- and distortion-corrected diffusion images.

Whole-brain tractography was computed from the preprocessed DWI by PyAFQ. Constrained spherical deconvolution was used to derive the fiber orientation distribution function (fODF) of each voxel, where constrained-positivity regularization = 1, minimum amplitude $\tau$ = 0.1, mean gray matter diffusivity = 0.0008, mean CSF diffusivity = 0.003, 600 fODF iterations, and spherical harmonics order = 8. Resulting fODFs of each voxel were then utilized to probabilistically generate fiber maps, using one seed per voxel for each dimension, a maximum turning angle of 30$^\circ$, step size = 0.5 mm, and a length range = 50-250 mm. The resulting fibers were parcellated into individual tracts via \textit{a priori} inclusion (waypoint) and exclusion regions of interest \parencite{wakana2007ReproducibilityQuantitativeTractography}. These tracts were then compared to a fiber probability map \parencite{hua2008TractProbabilityMaps} and any fibers which traverse low-probability spaces were removed from the tract. Any fibers with a length 3+ standard deviations from the tract average, or 4+ standard deviations from the average path centroid, were removed as well. Lastly, each tract was then resampled into 100 equidistant nodes (according to a Mahalanobis distance metric) from which averaged diffusion scalars (FA, AD, RD, and MD) were calculated. It was determined upon review of the 28 parcellated tract bundles that bilateral posterior arcuate and vertical occipital tracts were not well identified across all subjects and visits, accordingly analyses only included the remaining 24 white matter tracts. Finally, as scalar values approach zero at the start and end of tracts due to fiber fanning, fitting the distribution becomes rather problematic. We removed the first and last 10 nodes and were subsequently able to fit the data well, and we note that this clipping of the ends is in addition to that already performed by the PyAFQ software.


\subsection{Hierarchical Generalized Additive Models Fit Group Tract Scalars}
\label{ssec:meth-gam}
Hierarchical generalized additive models (HGAMs; \cite{pedersen2019HierarchicalGeneralizedAdditive}) allow for model fits at both global and group levels. That is, it is possible to model both the X-Y relationship that is shared across all levels of a factor (global smooth) and differences that factor levels (group smooths) may have from the global smooth. Further, it is not required that each level of smooth (global, group) contain the same `wiggliness' in the X-Y relationships. Separate smooth curves and wiggliness terms at different factor levels of HGAMs is highly relevant in modeling concussion-related changes within white matter tracts, as the global smooth of the tractometric profile (i.e. scalar values across all nodes) can effectively be held constant when modeling potential changes across visit, and independent wiggliness terms may capture scalar changes unique to one visit. Further, tensor product interaction terms can be used to build multimodal models, investigating the relationship of the tractometric profile with independent metrics such as the ImPACT composite scores. Accordingly, such a model would be capable not only of detecting changes within a tract that result from concussion, but also how such changes relate to clinical assessments. Finally, and critically, HGAMs facilitate conducting longitudinal, whole-brain analyses on tractometric profiles as data from all tracts and across all visits can be included in the same model. Such a specification allows for within-subject pooling of variance across both tract and visit. Where modeling individual tracts results in a creeping Type-I error and the corresponding corrections, concussion (and subsequent recovery) may affect multiple tracts within a subject and such shared variance would be lost when investigating tracts individually. By including all tracts and visits, HGAMs have the capability to not only reduce Type-I but also Type-II errors.

Interpreting GAM output involves considering both the significance and magnitude of the effect, as with the larger family of general linear models. GAMs test against an $H_0$ of flatness, or, whether the inclusion of wiggliness (more effective degrees of freedom) allow for a better fit the data. While smooths are penalized in order to resist over-fitting the data, subtle deflection from flatness can result in statistical, but not necessarily practical significance, and to disentangle this the relative magnitude of the partial effect (distance of the smooth from flat) is considered \parencite{baayen2020IntroductionGeneralizedAdditive}. We also note that the smooth's `confidence intervals' are actually Bayesian posterior credible intervals \parencite{pedersen2019HierarchicalGeneralizedAdditive}, and accordingly we can interpret smooths both with respect to zero and other smooths. When reporting results below, first the statistics are provided for the smooths of interest, which yield information about statistical deflections from flatness anywhere in the smooth. Second, plots will visualize the smooth and credible intervals, allowing interrogation of the partial effect with respect to magnitude, and to this end we plot group smooths in the same range as global smooths so the relative contribution of the partial effect to model fit can be captured. All GAMs were specified using the \lstinline{mgcv} package version 1.9-1 \parencite{wood2017GeneralizedAdditiveModels} in R version 4.3.3 \parencite{rcoreteam2023LanguageEnvironmentStatistical}.


\subsubsection{Whole Brain Longitudinal Difference Model}
\label{sssec:meth-gam-ldi}
To investigate within-tract concussion injury- and recovery-related FA changes we specified an HGAM to test for Post and RTP tract FA differences from Base. First, we calculated the Post-Base and RTP-Base changes in FA ($\Delta$FA); although including original FA values would be ideal, propagating ordered factors (Base $<$ Post $<$ RTP) across an interaction with another factor (tract) loses the original ordered structure. Here, ordered factors were necessary to investigate differences from baseline instead of merely the interaction with visit. $\Delta$FA values were modeled as a function of tract node using thin-plate regression splines (R Code \ref{code:gam-ldi}) and a basis dimensionality of 15 was determined sufficient to fit the tract curves (\lstinline{gam.check(fit_LDI)}). Subjects were treated as a random effect, thereby allowing each subject to have their own intercept across all levels of the factors, the $\Delta$FA distribution was well-fit by a Gaussian distribution with an identity link function, fast Residual Error of Maximum Likelihood (fREML) served as the smoothing parameter estimation method, and 12 threads were used in the computation (run time $\approx$ 45 minutes). Input data consisted of the 24 tracts with good segmentation across all subjects. Notably, we did not include a global smooth for this model, as the $\Delta$FA profile would differ for each tract, and we specified that each tract would have its own wiggliness term; essentially this is a longitudinal model of FA differences which references model `I' in \textcite{pedersen2019HierarchicalGeneralizedAdditive}.

\begin{equ}[H]
	\begin{lstlisting}
		fit_LDI <- mgcv::bam(
		  delta_fa ~ s(subj_id, by=tract_scan, bs="re") +
		    s(node_id, by=tract_scan, bs="tp", k=15) +
		    tract_name+visit_comp+tract_scan,
		  data=df,
		  family=gaussian(),
		  method="fREML",
		  nthreads=12
		)
	\end{lstlisting}
	\caption{$\Delta$FA values are modeled as a function of tract node with thin-plate regression smooths for each tract, accounting for the within-subject factors of tract and visit and using separate wiggliness terms for each tract. \lstinline{delta_fa} = RTP-Base and Post-Base FA differences, \lstinline{subj_id} = subject identifier factor, \lstinline{node_id} = node identifier integer, \lstinline{tract_name} = tract identifier factor, \lstinline{visit_comp} = visit comparison factor (RTP-Base, Post-Base), and \lstinline{tract_scan} = interaction of \lstinline{tract_name} and \lstinline{visit_comp}.}
	\label{code:gam-ldi}
\end{equ}


\subsubsection{Tract Longitudinal Scalar Model}
\label{sssec:meth-gam-lgio}
The model specified in R Code \ref{code:gam-ldi} effectively models the entire longitudinal dataset of $\Delta$FA values, allowing for pooling for variance within a subject across tract and visit, not requiring a multiple comparison correction for modeling all tracts. But as the $\Delta$FA calculation required participants to have data at Post and/or RTP in addition to Base (see Table \ref{tbl:meth-demo}), the analysis could not use all available data. As essentially a post-hoc analysis to further interrogate tract differences across visit, and also what change in scalar (e.g. increased RD) drove the difference in FA, individual tracts were modeled with a longitudinal HGAM with terms for global and group smooths (R Code \ref{code:gam-lgio}). Tract FA values were fit by a beta distribution with a logit link function, AD and RD values were fit with a Gaussian distribution and identity link function, and a gamma distribution with a logit link function fit the MD values. Subjects were again treated as a random effect, with separate intercepts for each scan (Base, Post, RTP), group smooths were allowed their own wiggliness parameter, and the colinearity of global and group smooths was controlled by the `m' parameter. Such a model is similar to model `GI' in \textcite{pedersen2019HierarchicalGeneralizedAdditive}. Finally, an ordered visit factor was included to test for difference in Post and RTP scalar values from Base. Such a model is particularly useful as the test statistic, which describes the flatness of the smooth, provides information about changes from Base values rather than deflections from zero. A model which fits group smooths for each visit is provided in Supplemental Materials (Supplemental R code \ref{supp-code:gam-lgi}).

\begin{equ}[H]
	\begin{lstlisting}
		df$scanOF <- factor(df$scan_name, ordered=T)
		fit_LGIO <- mgcv::bam(
		  <scalar> ~ s(subj_id, scan_name, bs="re") +
		    s(node_id, bs="tp", k=15, m=2) +
		    s(node_id, by=scanOF, bs="tp", k=15, m=1),
		  data=df,
		  family=<family>,
		  method="fREML",
		  nthreads=4
		)
	\end{lstlisting}
	\caption{Tract scalars are modeled as a function of tract node with thin-plate regression splines using both global and group (\lstinline{scan_name}) smooths as well as individual group wiggliness. An ordered factor of scan visit was used to compare Post and RTP to Base. \lstinline{<scalar>} = relevant DWI metric (AD, RD, MD, or FA), \lstinline{scan_name} = visit identifier factor (Base, Post, RTP), \lstinline{scanOf} = ordered factor of \lstinline{scan_name}, \lstinline{<family>} = relevant family and link function for scalar distribution.}
	\label{code:gam-lgio}
\end{equ}


\subsubsection{Tract Longitudinal Scalar Interaction Model}
\label{sssec:meth-gam-lgio-intx}
As noted above, GAMs are capable of modeling higher-dimensional, non-linear interactions through tensor product interaction smooths and hypersurfaces, a property which makes them particularly relevant for multimodal research. We used such a model to test whether concussion injury- and recovery-related changes in tract scalars related to changes in ImPACT composite and total symptom scores (R code \ref{code:gam-lgio-intx}), thereby potentially linking damage within a specific region of a tract to changes in assessment metrics. Tract scalars were modeled as a function of both tract node and ImPACT measure, and the node-ImPACT interaction term was specified such that each visit (Base, Post, RTP) would have a different scalar-node-ImpACT interaction surface. We note the decrease in basis dimensionality for the ImPACT measures thin-plate regression splines from the default value, and that fitting the tensor product interaction smooth also benefited from a slightly higher basis dimensions term for the tract node term. Additionally, and as above (Section \ref{sssec:meth-gam-lgio}), an ordered factor for visit was included to test whether the Post and RTP interaction surfaces differed from that of Base; a model with separate interaction surfaces is provided in Supplemental Materials (Supplemental R code \ref{supp-code:gam-lgi-intx}).

\begin{equ}[H]
	\begin{lstlisting}
		df$scanOF <- factor(df$scan_name, ordered=T)
		fit_LGIO_intx <- mgcv::bam(
		  <scalar> ~ s(subj_id, scan_name, bs="re") +
		    s(node_id, bs="tp", k=15, m=2) +
		    s(imp_meas, by=scan_name, bs="tp", k=5) +
		    ti(node_id, imp_meas, bs=c"tp","tp"), k=c(20,5), m=1) +
		    ti(
		      node_id, imp_meas, by=scanOF,
		      bs=c("tp","tp"), k=c(20,5), m=1
		    ),
		  data=df,
		  family=<family>,
		  method="fREML",
		  nthreads=4
		)
	\end{lstlisting}
	\caption{Tract scalars are modeled as a function of separate 1D node and ImPACT smooths as well as a 2D tensor product interaction surface, with ordered factors used to compare Post and RTP surfaces to Base. \lstinline{imp_meas} = ImPACT composite or total symptom measure.}
	\label{code:gam-lgio-intx}
\end{equ}

Finally, another potentially powerful use of tensor product interaction smooths is to model changes in tract scalars as a function of time. Given the within-tract temporal dynamics of axonal injury and recovery, the ability to model non-linear temporal interactions with scalar changes will help quantify the progression of injury. For such an analysis, however, multiple scans would be required of each participant to effectively track signal associated with recovery. As our data were collected only at Post and RTP, we modeled tract FA differences between Post and RTP (RTP-Post $\Delta$FA) as a function of node and number of days between the two visits (via a model similar to Supplemental R code \ref{supp-code:gam-lgi-intx}). Such an analysis has the potential to identify which tract changes are associated with the longest recovery periods, as presumably less-severe injuries are associated with fewer days between Post and RTP. We also note that few participants had recovery periods longer than 14 days, so any detected interactions at longer recovery periods may be driven by relatively few participants (recovery duration = 9$\pm$5.5 days, recovery duration $>$14 days = 6 participants; 1 outlier omitted (93 days) due to poor model fit).

\subsubsection{ImPACT model}
\label{sssec:meth-gam-impact}
The relationship between visit (Base, Post, RTP), ImPACT composite metrics (verbal memory, visual memory, visual motor, impulse control, and reaction time), and ImPACT total symptom scores were modeled with GAMs to test for changes across assessment visit. As with tract scalar profiles, GAMs were employed as (a) non-linear trends are expected in such metrics and (b) they can model the semi-parametric distributions encountered in several of the metrics. Each ImPACT metric was fit as a function of assessment number, using integer values rather than categorical Base, Post, and RTP (Supplemental R Code \ref{supp-code:gam-impact}); such a specification allowed for modeling evolving changes in assessment metrics rather than comparing main effects across factor levels. Verbal and visual memory composites were converted to proportion scores and modeled with a beta distribution and logit link function, visual motor speed and reaction time were best fit with Gaussian distributions and identity link functions (despite the skewness), and a negative binomial distribution with log link function fit the impulse control and total symptoms well.

When specifying models, whether with ImPACT or DWI data, model fits were reviewed and assessed via \lstinline{mgcv:gam.check()}, and the selection of competing models was aided by \lstinline{itsadug::compareML()}. Pipeline and statistical code are available at the project repository: \url{https://github.com/nmuncy/adr_dwi}.


\section{Results}
\label{sec:res}

\subsection{ImPACT Visual Memory, Reaction Time, and Total Symptoms Show Evidence of Injury and Recovery}
\label{ssec:res-imp}
All models of ImPACT metrics (Section \ref{sssec:meth-gam-impact}), except for impulse control, detected a sig\-nificant interaction between the ImPACT metric and assessment number (Figure \ref{fig:imp-gam}). Visual memory, reaction time, and total symptoms demonstrated patterns consistent with concussion-related deficits at Post and subsequent recovery at RTP (visual memory: $F_{(1.94, 1.99)}$ = 8.59, \textit{p} $<$ .001; reaction time: $F_{(1.91, 1.99)}$ = 6.18, \textit{p} $<$ .01; total symptoms: $F_{(1.98, 1.99)}$ = 28.74, \textit{p} $<$ .0001), and we note that total symptoms at RTP were much lower than at Base (Figure \ref{fig:imp-gam}, bottom right).

Conversely, while verbal memory and visual motor tests indicate significant non-flatness (verbal memory: $F_{(1.82, 1.96)}$ = 4.34, \textit{p} = .028; visual motor: $F_{(1.86, 1.97)}$ = 8.19, \textit{p} $<$ .001), concussion-related changes were not detected between Base and Post and the statistic is driven by increased performance at RTP. This pattern possibly reflects a lack of sensitivity at Base and/or practice effects. Finally, impulse control was unchanged (i.e. flat) as a function of assessment ($F_{(1.0, 1)}$ = .003, \textit{p} = .95).

\begin{figure}[H]
	\centering
	\fbox{\includegraphics{fig_impact_gam.png}}
	\caption{GAM smooths for ImPACT composite and total symptom scores. Assessment numbers where the confidence interval does not include 0 indicate significant changes. Visual memory, reaction time, and total symptoms showed worsening and then recovery (U-shapes) while verbal memory and visual motor speed scores were better at assessment 3. Impulse control did not change across visits. Visit 1=Base, 2=Post, 3=RTP.}
	\label{fig:imp-gam}
\end{figure}


\subsection{Concussion and Recovery in DWI Tractometric Profiles}
\label{ssec:res-dwi-tract}

\subsubsection{Longitudinal Whole-Brain Analyses Implicate Injury in Multiple Tracts}
\label{sssec:res-dwi-tract-wba}
The longitudinal, whole-brain difference model (Section \ref{sssec:meth-gam-ldi}) produced tract difference smooths for Post-Base and RTP-Base $\Delta$FA values (Figure \ref{fig:ldi-gam}, top and middle). Surprisingly, test statistics for all smooths indicated significant non-flatness, suggesting that at least some regions of each tract differed significantly between Base and both Post and RTP (Table \ref{tbl:ldi-gam}). We note that in interpreting GAM coefficients, the magnitude of the effect is equally relevant to the test statistic of non-flatness (F-stat). For instance, when interpreting Figure \ref{fig:ldi-gam}, the callosum orbital (left, green) had a much larger magnitude compared to another, equally `significant' tract (callosum temporal, pink). Nevertheless, it is well established that concussion is associated with negative diagnostic readings \parencite[e.g.][]{klein2019PrevalencePotentiallyClinically}, and we expected to only find changes to scalar values in regions commonly associated with traumatic axonal injury e.g. tracts which decussate through the splenium (here the superior parietal and posterior parietal callosal tracts).

\begin{figure}[H]
	\centering
	\fbox{\includegraphics[width=0.95\textwidth]{fig_LDI_DI_rerun.png}}
	\caption{HGAM smooths of tract FA differences by node for comparisons against Base (top, middle) and Run-Rerun (bottom). \textbf{Top}: $\Delta$FA-Node smooths of Post relative to Base, deflections from zero here may be indicative of concussion-related changes. \textbf{Middle}: $\Delta$FA-Node smooths of RTP relative to Base, smaller deflections here than Post vs. Base may be indicative of recovery-related changes (and the converse for greater deflections). \textbf{Bottom}: $\Delta$FA-Node smooths of Run-Rerun, indicative of algorithmic variance by tract. Node IDs are numbered in the RAS convention, with midline of commissural fibers at node 50. Left: Corpus callosum tracts, middle: left hemisphere tracts, right: right hemisphere tracts. Confidence intervals were omitted for visual clarity. \textit{Commissural tracts}: CCaf = anterior forceps, CCmot = motor, CCocc = occipital, CCorb = orbitalis, CCpp = posterior parietal, CCsf = superior frontal, CCsp = superior parietal, CCtemp = temporal. \textit{Association tracts}: Arc = arcuate, aThal = anterior thalamic, CCing = cingulum (cingulate portion), CS = corticospinal, IFO = inferior fronto-occipital, IL = inferior lateral, SL = superior lateral, Unc = uncinate; association tract acronyms are preceded by `l'/`r' to designate left/right hemisphere. High-resolution figure available online.}
	\label{fig:ldi-gam}
\end{figure}

% \begin{Definition}{}{}
% 	Definitions of acronyms used for PyAFQ tracts.\\

% 	\underline{Commissural Tracts}\\
% 	CCaf: corpus callosum, anterior forceps portion\\
% 	CCmot: corpus callosum, motor portion\\
% 	CCocc: corpus callosum, occipital portion\\
% 	CCorb: corpus callosum, orbitalis portion\\
% 	CCpp: corpus callosum, posterior parietal portion\\
% 	CCsf: corpus callosum, superior frontal portion\\
% 	CCsp: corpus callosum, superior parietal portion\\
% 	CCtemp: corpus callosum, temporal portion\\

% 	\underline{Association Tracts}\\
% 	Arc: arcuate fasciculus\\
% 	aThal: anterior thalamic fibers\\
% 	CCing: cingulum, cingulate portion\\
% 	CS: corticospinal tract\\
% 	IFO: inferior fronto-occipital fasciculus\\
% 	IL: inferior lateral fasciculus\\
% 	SL: superior lateral fasciculus\\
% 	Unc: uncinate fasciculus

% 	\textbf{Note:} association tract acronyms are preceded by `l'/`r' to designate left/right hemisphere.
% 	\label{box:tract-names}
% \end{Definition}


As tractometry was conducted on each visit via PyAFQ, one potential source of variance is algorithmic (i.e. pipeline) run-rerun (test-retest) stability \parencite{wang2011LongitudinalChangesStructural}, particularly given our use of probabilistic tractography. Previous work has demonstrated high test-retest reliability metrics for PyAFQ \parencite{kruper2021EvaluatingReliabilityHuman}, which also noted tract-dependent reliability metrics that averaged around 86\%. We note that concussion-related scalar changes may not be as large as 14\%, particularly at the group level due to injury heterogeneity. Nevertheless, we reran tractography on the Base visit and calculated Run-Rerun $\Delta$FA values for each subject's tract profiles to quantify the amount of algorithmic variance in our data. The Run-Rerun $\Delta$FA were modeled with a non-longitudinal variant of R Code \ref{code:gam-ldi} (Supplemental R Code \ref{supp-code:gam-di}). Resulting smooths (Figure \ref{fig:ldi-gam}, bottom) illustrate the amount of algorithmic variance that can be expected for each tract, where some tracts have demonstrably high variance (e.g. callosum orbitalis) while others are more stable (superior frontal callosum). Corresponding test statistics identified a number of tracts which did not significantly differ between run and rerun (Table \ref{tbl:ldi-gam}).

\begin{table}[H]
	\scriptsize
	% Please add the following required packages to your document preamble:
% \usepackage[table,xcdraw]{xcolor}
% Beamer presentation requires \usepackage{colortbl} instead of \usepackage[table,xcdraw]{xcolor}

\begin{tabular}{llll|llllll}
 & \multicolumn{3}{c|}{Post-Base} & \multicolumn{3}{c|}{RTP-Base} & \multicolumn{3}{c}{Run-Rerun} \\ \cline{2-10}
Tract & \multicolumn{1}{c}{edf} & \multicolumn{1}{c}{F} & \multicolumn{1}{c|}{Sig} & \multicolumn{1}{c}{edf} & \multicolumn{1}{c}{F} & \multicolumn{1}{c|}{Sig} & \multicolumn{1}{c}{edf} & \multicolumn{1}{c}{F} & \multicolumn{1}{c}{Sig} \\ \hline
\multicolumn{1}{l|}{CCaf} & 4.58 & 5.52 & \textbf{***} & 5.82 & 11.90 & \multicolumn{1}{l|}{\textbf{***}} & 1.00 & 0.34 & \textbf{} \\
\rowcolor[HTML]{C0C0C0}
\multicolumn{1}{l|}{\cellcolor[HTML]{C0C0C0}CCmot} & 9.63 & 9.89 & \textbf{***} & 4.34 & 15.23 & \multicolumn{1}{l|}{\cellcolor[HTML]{C0C0C0}\textbf{***}} & 2.27 & 1.84 & \textbf{} \\
\multicolumn{1}{l|}{CCocc} & 7.13 & 9.12 & \textbf{***} & 6.79 & 9.53 & \multicolumn{1}{l|}{\textbf{***}} & 1.00 & 0.00 & \textbf{} \\
\rowcolor[HTML]{C0C0C0}
\multicolumn{1}{l|}{\cellcolor[HTML]{C0C0C0}CCorb} & 9.71 & 30.92 & \textbf{***} & 10.56 & 28.06 & \multicolumn{1}{l|}{\cellcolor[HTML]{C0C0C0}\textbf{***}} & 12.01 & 36.29 & \textbf{***} \\
\multicolumn{1}{l|}{CCpp} & 4.35 & 7.62 & \textbf{***} & 3.78 & 9.10 & \multicolumn{1}{l|}{\textbf{***}} & 8.01 & 4.43 & \textbf{***} \\
\rowcolor[HTML]{C0C0C0}
\multicolumn{1}{l|}{\cellcolor[HTML]{C0C0C0}CCsf} & 7.20 & 9.11 & \textbf{***} & 3.15 & 4.77 & \multicolumn{1}{l|}{\cellcolor[HTML]{C0C0C0}\textbf{***}} & 1.00 & 3.95 & \textbf{*} \\
\multicolumn{1}{l|}{CCsp} & 7.09 & 20.93 & \textbf{***} & 4.75 & 21.22 & \multicolumn{1}{l|}{\textbf{***}} & 1.43 & 0.34 & \textbf{} \\
\rowcolor[HTML]{C0C0C0}
\multicolumn{1}{l|}{\cellcolor[HTML]{C0C0C0}CCtemp} & 6.40 & 6.61 & \textbf{***} & 6.32 & 4.43 & \multicolumn{1}{l|}{\cellcolor[HTML]{C0C0C0}\textbf{***}} & 12.31 & 6.23 & \textbf{***} \\
\multicolumn{1}{l|}{laThal} & 9.98 & 12.34 & \textbf{***} & 10.55 & 15.29 & \multicolumn{1}{l|}{\textbf{***}} & 8.20 & 8.54 & \textbf{***} \\
\rowcolor[HTML]{C0C0C0}
\multicolumn{1}{l|}{\cellcolor[HTML]{C0C0C0}lArc} & 8.59 & 2.67 & \textbf{**} & 9.78 & 6.50 & \multicolumn{1}{l|}{\cellcolor[HTML]{C0C0C0}\textbf{***}} & 1.26 & 0.06 & \textbf{} \\
\multicolumn{1}{l|}{lCCing} & 7.17 & 15.00 & \textbf{***} & 6.96 & 6.98 & \multicolumn{1}{l|}{\textbf{***}} & 1.00 & 70.30 & \textbf{***} \\
\rowcolor[HTML]{C0C0C0}
\multicolumn{1}{l|}{\cellcolor[HTML]{C0C0C0}lCS} & 6.41 & 37.00 & \textbf{***} & 8.88 & 15.25 & \multicolumn{1}{l|}{\cellcolor[HTML]{C0C0C0}\textbf{***}} & 1.00 & 1.82 & \textbf{} \\
\multicolumn{1}{l|}{lIFO} & 10.64 & 18.33 & \textbf{***} & 9.48 & 13.52 & \multicolumn{1}{l|}{\textbf{***}} & 7.20 & 7.73 & \textbf{***} \\
\rowcolor[HTML]{C0C0C0}
\multicolumn{1}{l|}{\cellcolor[HTML]{C0C0C0}lIL} & 3.38 & 7.15 & \textbf{***} & 1.01 & 11.67 & \multicolumn{1}{l|}{\cellcolor[HTML]{C0C0C0}\textbf{***}} & 7.83 & 4.46 & \textbf{***} \\
\multicolumn{1}{l|}{lSL} & 6.58 & 8.14 & \textbf{***} & 1.00 & 7.67 & \multicolumn{1}{l|}{\textbf{***}} & 1.61 & 0.88 & \textbf{} \\
\rowcolor[HTML]{C0C0C0}
\multicolumn{1}{l|}{\cellcolor[HTML]{C0C0C0}lUnc} & 6.45 & 10.78 & \textbf{***} & 2.00 & 10.26 & \multicolumn{1}{l|}{\cellcolor[HTML]{C0C0C0}\textbf{***}} & 8.57 & 15.05 & \textbf{***} \\
\multicolumn{1}{l|}{raThal} & 7.32 & 12.45 & \textbf{***} & 9.06 & 17.64 & \multicolumn{1}{l|}{\textbf{***}} & 5.68 & 9.47 & \textbf{***} \\
\rowcolor[HTML]{C0C0C0}
\multicolumn{1}{l|}{\cellcolor[HTML]{C0C0C0}rArc} & 7.33 & 17.97 & \textbf{***} & 6.94 & 7.59 & \multicolumn{1}{l|}{\cellcolor[HTML]{C0C0C0}\textbf{***}} & 5.41 & 5.30 & \textbf{***} \\
\multicolumn{1}{l|}{rCCing} & 9.41 & 18.11 & \textbf{***} & 6.07 & 11.80 & \multicolumn{1}{l|}{\textbf{***}} & 10.61 & 14.86 & \textbf{***} \\
\rowcolor[HTML]{C0C0C0}
\multicolumn{1}{l|}{\cellcolor[HTML]{C0C0C0}rCS} & 8.95 & 6.66 & \textbf{***} & 4.17 & 7.53 & \multicolumn{1}{l|}{\cellcolor[HTML]{C0C0C0}\textbf{***}} & 5.18 & 6.69 & \textbf{***} \\
\multicolumn{1}{l|}{rIFO} & 10.10 & 15.38 & \textbf{***} & 10.60 & 16.69 & \multicolumn{1}{l|}{\textbf{***}} & 11.49 & 8.77 & \textbf{***} \\
\rowcolor[HTML]{C0C0C0}
\multicolumn{1}{l|}{\cellcolor[HTML]{C0C0C0}rIL} & 5.74 & 12.47 & \textbf{***} & 4.99 & 11.56 & \multicolumn{1}{l|}{\cellcolor[HTML]{C0C0C0}\textbf{***}} & 6.18 & 5.82 & \textbf{***} \\
\multicolumn{1}{l|}{rSL} & 7.32 & 7.52 & \textbf{***} & 2.30 & 3.64 & \multicolumn{1}{l|}{\textbf{*}} & 7.45 & 2.92 & \textbf{**} \\
\rowcolor[HTML]{C0C0C0}
\multicolumn{1}{l|}{\cellcolor[HTML]{C0C0C0}rUnc} & 8.43 & 11.34 & \textbf{***} & 10.18 & 18.80 & \multicolumn{1}{l|}{\cellcolor[HTML]{C0C0C0}\textbf{***}} & 8.62 & 16.32 & \textbf{***}
\end{tabular}

	\caption{Longitudinal whole-brain HGAM statistics for tract smooths. Significant non-flatness was detected for all difference smooths in the longitudinal concussion model (Post-Base, RTP-Base), and a number of tracts did not show significant differences between multiple runs of the tractography pipeline (Run-Rerun) e.g. CCsp. Post-Base = FA difference between Post and Base, RTP-Base = FA difference between RTP and Base, Run-Rerun = FA difference between multiple runs of PyAFQ. edf = effective degrees of freedom, F = F-statistic, Sig = significance. *** = p$<$.001, ** = p$<$.01, * = p$<$.05.}
	\label{tbl:ldi-gam}
\end{table}

Together, these whole-brain longitudinal and Run-Rerun analyses of tract FA changes offer support of our first hypothesis that concussion-related changes would be detected in the posterior callosum. Significant Post-Base and RTP-Base differences were detected in the superior posterior region of corpus callosum (CCsp) by the longitudinal concussion HGAM while, importantly, such differences were not driven by algorithmic variance. Further, non-trivial difference magnitudes above and beyond any potential algorithmic variance were detected in a number of callosal and ipsilateral tracts, discussed below.


\subsubsection{Tract-Specific Analyses Determine Source of FA Changes}
\label{sssec:res-dwi-tract-tsa}
\textit{Corpus Callosum Results}. Close inspection of the whole-brain longitudinal and run-rerun results implicated four callosal tracts that demonstrated concussion-related statistics that were practically (cf. statistically) significant: superior parietal, superior frontal, motor, and orbital. Each of these tracts had non-trivial partial effect magnitudes as well as $\Delta$FA that were greater than was explained by algorithmic variance alone. As FA differences can be driven by changes in AD \textit{or} RD, which are related to different injury sequelae, we modeled FA, MD, AD, and RD for each of these tracts, testing for Post and RTP differences from Base (R Code \ref{code:gam-lgio}). This analysis also had an additional benefit: where $\Delta$FA calculations require data at both visits for subtraction, these longitudinal models of a single tract and scalar had a reduced factor structure such that we were able to use scalars as the predicted values, thereby increasing the number of participants in the model (see Table \ref{tbl:meth-demo}).

Statistical significance was detected for each scalar's smooths (Table \ref{tbl:lgio-gam-cc}), and inspection of these smooths revealed three patterns of injury (Figure \ref{fig:lgio-gam-cc}). First, the Post vs Base and RTP vs Base FA differences appear to be constant for the superior parietal tract (Figure \ref{fig:lgio-gam-cc}, A), where FA values about nodes 50-60 are increased relative to Base and this FA difference is seemingly driven by a decreased RD. Such a pattern is consistent with cellular edema, particularly given the lack of change in AD. Second, evidence of recovery is apparent in the superior frontal tract (Figure \ref{fig:lgio-gam-cc}, B), where FA values about node 50 are elevated at Post relative to Base, and RD values decreased, a pattern that is again consistent with cellular edema. Both scalars return to near-Base values at RTP, seemingly indicative of healing.

\begin{table}[H]
	\scriptsize
	% Please add the following required packages to your document preamble:
% \usepackage{multirow}
% \usepackage[table,xcdraw]{xcolor}
% Beamer presentation requires \usepackage{colortbl} instead of \usepackage[table,xcdraw]{xcolor}

\begin{tabular}{llll|ll|ll|ll}
 &  & \multicolumn{2}{c|}{FA} & \multicolumn{2}{c|}{MD} & \multicolumn{2}{c|}{AD} & \multicolumn{2}{c}{RD} \\ \cline{3-10}
Tract & Smooth & \multicolumn{1}{c}{edf} & \multicolumn{1}{c|}{F} & \multicolumn{1}{c}{edf} & \multicolumn{1}{c|}{F} & \multicolumn{1}{c}{edf} & \multicolumn{1}{c|}{F} & \multicolumn{1}{c}{edf} & \multicolumn{1}{c}{F} \\ \hline
 & \multicolumn{1}{l|}{Global} & 13.96 & 2750.34 & 13.29 & 256.90 & 13.96 & 4077.09 & 13.83 & 636.96 \\
 & \multicolumn{1}{l|}{\cellcolor[HTML]{C0C0C0}O.Post} & \cellcolor[HTML]{C0C0C0}8.21 & \cellcolor[HTML]{C0C0C0}9.66 & \cellcolor[HTML]{C0C0C0}9.46 & \cellcolor[HTML]{C0C0C0}12.01 & \cellcolor[HTML]{C0C0C0}6.94 & \cellcolor[HTML]{C0C0C0}3.76 & \cellcolor[HTML]{C0C0C0}8.95 & \cellcolor[HTML]{C0C0C0}11.00 \\
\multirow{-3}{*}{CCsp} & \multicolumn{1}{l|}{O.RTP} & 8.10 & 11.02 & 9.30 & 12.57 & 6.29 & 3.02 & 9.04 & 11.67 \\
\rowcolor[HTML]{EFEFEF}
\cellcolor[HTML]{EFEFEF} & \multicolumn{1}{l|}{\cellcolor[HTML]{EFEFEF}Global} & 13.95 & 582.48 & 13.80 & 2762.05 & 13.97 & 4854.79 & 13.82 & 513.35 \\
\rowcolor[HTML]{C0C0C0}
\cellcolor[HTML]{EFEFEF} & \multicolumn{1}{l|}{\cellcolor[HTML]{C0C0C0}O.Post} & 8.99 & 4.70 & 6.23 & 2.68 & 6.29 & 1.14 & 8.53 & 5.78 \\
\rowcolor[HTML]{EFEFEF}
\multirow{-3}{*}{\cellcolor[HTML]{EFEFEF}CCsf} & \multicolumn{1}{l|}{\cellcolor[HTML]{EFEFEF}O.RTP} & 7.01 & 3.51 & 8.33 & 6.03 & 6.97 & 1.81 & 7.71 & 5.81 \\
\rowcolor[HTML]{FFFFFF}
\cellcolor[HTML]{FFFFFF} & \multicolumn{1}{l|}{\cellcolor[HTML]{FFFFFF}Global} & 13.95 & 538.33 & 13.86 & 3284.55 & 13.98 & 5804.04 & 13.62 & 501.66 \\
\rowcolor[HTML]{C0C0C0}
\cellcolor[HTML]{FFFFFF} & \multicolumn{1}{l|}{\cellcolor[HTML]{C0C0C0}O.Post} & 8.88 & 4.34 & 8.91 & 6.27 & 6.37 & 2.00 & 8.91 & 4.87 \\
\rowcolor[HTML]{FFFFFF}
\multirow{-3}{*}{\cellcolor[HTML]{FFFFFF}CCmot} & \multicolumn{1}{l|}{\cellcolor[HTML]{FFFFFF}O.RTP} & 8.41 & 8.05 & 8.72 & 8.74 & 6.41 & 1.26 & 8.28 & 8.54 \\
\rowcolor[HTML]{EFEFEF}
\cellcolor[HTML]{EFEFEF} & \multicolumn{1}{l|}{\cellcolor[HTML]{EFEFEF}Global} & 13.01 & 500.23 & 10.65 & 94.12 & 12.52 & 549.89 & 12.27 & 94.55 \\
\rowcolor[HTML]{C0C0C0}
\cellcolor[HTML]{EFEFEF} & \multicolumn{1}{l|}{\cellcolor[HTML]{C0C0C0}O.Post} & 10.48 & 9.31 & 7.15 & 3.24 & 8.15 & 3.00 & 9.13 & 6.58 \\
\rowcolor[HTML]{EFEFEF}
\multirow{-3}{*}{\cellcolor[HTML]{EFEFEF}CCorb} & \multicolumn{1}{l|}{\cellcolor[HTML]{EFEFEF}O.RTP} & 11.47 & 12.39 & 11.92 & 16.88 & 10.86 & 11.00 & 11.95 & 16.50
\end{tabular}

	\caption{Tract-specific HGAM statistics for DWI scalars of select tracts. Separate models were conducted for each scalar of each tract, fitting both the Global curvature and Group (Post, RTP) differences from Base. O.Post/RTP = Post/RTP group smooth as an ordered factor (relative to Base). edf = effective degrees of freedom, F = F-statistic, Sig = significance. *** = p$<$.001, ** = p$<$.01, * = p$<$.05.}
	\label{tbl:lgio-gam-cc}
\end{table}

Finally, two tracts appear to worsen between Post and RTP: the motor and orbital tracts (Figure \ref{fig:lgio-gam-cc}, C \& D). As with the other tracts, the motor portion of corpus callosum appears to have inflated FA values relative to Base about node 50, driven by decreased RD. Instead of constancy or recovery, however, the difference in scalar values spreads leftward (nodes 40-50). Conversely, the orbitalis portion has a decrease in FA values around node 60 (and increased RD) in Post relative to Base, a difference which increases in magnitude by RTP. This pattern is likely reflective of axolemmal permeability or demyelination rather than cellular or cytotoxic edema.

\begin{figure}[H]
	\centering
	\fbox{\includegraphics[width=0.95\textwidth]{fig_LGIO_callosal_smooths.png}}
	\caption{Longitudinal HGAM smooths modeling callosal tract scalars as a function of node. Each model fit both the global tractometric profile (global smooths) and how visits differed from the global (group smooths). Group smooths were modeled using ordered factors (Post vs Base, RTP vs Base).  Each quadrant (A-D) contains global and group smooths for FA (top left), MD (top right), AD (bottom left), and RD (bottom right). Red boxes indicate nodes where smooths differ statistically from the reference group (Base). Group smooths are plotted in the domain of the global smooth to show their fit contribution. High-resolution figure available online.}
	\label{fig:lgio-gam-cc}
\end{figure}

These patterns of concussion injury- and recovery-related scalar changes partially confirms our second hypothesis that FA decreases would be driven by increased RD. While the expected directionality of injury-related scalar changes was only detected in the orbital tract (Figure \ref{fig:lgio-gam-cc}, D), we nevertheless observed that changes in FA were largely driven by RD ($\lambda_\perp$) and not AD ($\lambda_\parallel$).


\textit{Select Tract Results}. In addition to those of the corpus callosum, we identified a subset of tracts that showed concussion injury- or recovery-related scalar changes with magnitudes larger than were expected from algorithmic variance: left arcuate, left corticospinal, right anterior thalamic, right cingulum cingulate, right inferior fronto-occipital, and right uncinate (Figure \ref{fig:lgio-gam-sel}). Test statistics (Table \ref{tbl:lgio-gam-cc}) indicated that all ordered group smooths differed significantly from Base in at least one region of nodes, except for the left arcuate Post FA and left corticospinal Post AD smooths (Figure \ref{fig:lgio-gam-sel}, A \& B).

\begin{figure}[H]
	\centering
	\fbox{\includegraphics[width=0.95\textwidth]{fig_LGIO_select_smooths.png}}
	\caption{Longitudinal HGAM smooths modeling select tract scalars as a function of node. Each section (A-F) consists of global and visit smooths for FA (top left), MD (top right), AD (bottom left), and RD (bottom right). Red boxes indicate nodes where smooths differ statistically from the reference group (Base). Group smooths are plotted in the domain of the global smooth to show their fit contribution. High-resolution figure available online.}
	\label{fig:lgio-gam-sel}
\end{figure}

Within these selected tracts, as with the callosal tracts, patterns of injury stability, recovery, and worsening are present. First, the corticospinal, anterior thalamic, and inferior fronto-occipital appear to have stable FA changes relative to Base at Post and RTP. For the corticospinal, increases in FA (Figure \ref{fig:lgio-gam-sel}B, node 65) are driven by decreases in RD for the same region, and this pattern is also found for rIFO (Figure \ref{fig:lgio-gam-sel}E, node 45). For anterior thalamic, both decreases in RD and increases in AD are associated with the increase in FA values (Figure \ref{fig:lgio-gam-sel}C, node 65). Second, the arcuate and uncinate tracts both show evidence of worsening, where tract scalar become more discrepant with Base by RTP. At Post, the arcuate FA values do not differ from Base (albeit the increases in MD, AD, and RD; Figure \ref{fig:lgio-gam-sel}A), and by RTP a steep decrease in FA about node 65 is present which is driven by corresponding increase in RD. Likewise, the FA values for uncinate worsen over time (Figure \ref{fig:lgio-gam-sel}F), most dramatically in the increased FA values near node 70, an increase which is driven by RD decreases. Finally, the right cingulum cingulate (Figure \ref{fig:lgio-gam-sel}D) shows evidence of recovery near node 35. At Post, a steep decrease in FA is associated with a corresponding increase in RD, which diminishes by RTP. That said, the slight increase in FA about node 70 at Post seems to elongate by RTP, with a decline in RD as well.

Together, these results largely mirrored those detected in the callosal tracts: our hypothesized pattern of scalar change was detected in the right cingulum cingulate, and FA changes across all tracts were largely associated with RD. FA increases driven by RD decreases are commonly implicated in cellular or cytotoxic edema, while the reverse can be understood in terms of axolemmal permeability or demyelination. Finally, we note that the left arcuate showed a pattern of late emergent pathology.


\subsection{Tract FA Differences Relate to Changes in ImPACT Scores}
\label{ssec:res-dwi-imp}
\textit{Corpus Callosum Interaction Results}. Tensor product interaction smooths support multimodal interaction models. In addition to modeling within-tract longitudinal scalar changes, clinical assessment metrics (or data from other modalities) can be included in higher dimensional models (R Code \ref{code:gam-lgio-intx}) to investigate whether such assessment metrics relate to within-tract concussion-related scalar changes. We conducted exploratory analyses to determine whether changes in callosal FA values related to ImPACT composite and total symptom scores. Specifically, we tested whether the tract FA-ImPACT interaction at Post and RTP differed from those at Base, where Post differences may link physiologic changes (tract scalars) to clinical assessment and a lack of such differences at RTP may indicate recovery.

Of the eight callosal tracts tested, only three demonstrated the hypothesized interaction (Figure \ref{fig:intro-hyp}, right), where estimated FA values would relate to ImPACT scores within a specific region of the tract at Post but not RTP. Specifically, the superior frontal FA changes interacted with changes in both visual memory and total symptom scores, occipital FA changes interacted with visual motor, and superior parietal with total symptom scores. Test statistics (Table \ref{tbl:lgio-intx-cc}, top) indicate that while controlling for the curvature of the tract (Global) and the main interaction effect (ImPACT-Node), a significant differential interaction exists at Post relative to Base (ImPACT-Node:O.Post) that is also reduced at RTP (ImPACT-Node:O.RTP).

\begin{table}[H]
	\scriptsize
	% Please add the following required packages to your document preamble:
% \usepackage[table,xcdraw]{xcolor}
% Beamer presentation requires \usepackage{colortbl} instead of \usepackage[table,xcdraw]{xcolor}

\begin{tabular}{lllllllllllll}
 & \multicolumn{3}{c|}{CCsf: VisMem} & \multicolumn{3}{c|}{CCsf: TotSymp} & \multicolumn{3}{c|}{CCocc: VisMot} & \multicolumn{3}{c}{CCsp: TotSymp} \\ \cline{2-13}
Smooth & \multicolumn{1}{c}{edf} & \multicolumn{1}{c}{F} & \multicolumn{1}{c|}{Sig} & \multicolumn{1}{c}{edf} & \multicolumn{1}{c}{F} & \multicolumn{1}{c|}{Sig} & \multicolumn{1}{c}{edf} & \multicolumn{1}{c}{F} & \multicolumn{1}{c|}{Sig} & \multicolumn{1}{c}{edf} & \multicolumn{1}{c}{F} & \multicolumn{1}{c}{Sig} \\ \hline
\multicolumn{1}{l|}{Node} & 13.95 & 775.40 & \multicolumn{1}{l|}{***} & 13.93 & 527.98 & \multicolumn{1}{l|}{***} & 13.79 & 2681.43 & \multicolumn{1}{l|}{***} & 13.92 & 2378.29 & *** \\
\rowcolor[HTML]{C0C0C0}
\multicolumn{1}{l|}{\cellcolor[HTML]{C0C0C0}ImP:Base} & 1.00 & 0.92 & \multicolumn{1}{l|}{\cellcolor[HTML]{C0C0C0}} & 1.00 & 1.25 & \multicolumn{1}{l|}{\cellcolor[HTML]{C0C0C0}} & 1.76 & 1.12 & \multicolumn{1}{l|}{\cellcolor[HTML]{C0C0C0}} & 1.69 & 0.77 &  \\
\multicolumn{1}{l|}{ImP:Post} & 1.00 & 3.64 & \multicolumn{1}{l|}{} & 1.00 & 3.08 & \multicolumn{1}{l|}{} & 1.01 & 0.01 & \multicolumn{1}{l|}{} & 2.77 & 1.87 &  \\
\rowcolor[HTML]{C0C0C0}
\multicolumn{1}{l|}{\cellcolor[HTML]{C0C0C0}ImP:RTP} & 1.00 & 0.17 & \multicolumn{1}{l|}{\cellcolor[HTML]{C0C0C0}} & 1.34 & 0.25 & \multicolumn{1}{l|}{\cellcolor[HTML]{C0C0C0}} & 1.01 & 0.21 & \multicolumn{1}{l|}{\cellcolor[HTML]{C0C0C0}} & 1.35 & 0.31 &  \\
\multicolumn{1}{l|}{ImP-Node} & 37.14 & 1.93 & \multicolumn{1}{l|}{***} & 37.56 & 2.83 & \multicolumn{1}{l|}{***} & 61.19 & 7.50 & \multicolumn{1}{l|}{***} & 46.82 & 3.72 & *** \\
\rowcolor[HTML]{C0C0C0}
\multicolumn{1}{l|}{\cellcolor[HTML]{C0C0C0}ImP-Node:O.Post} & 51.17 & 7.09 & \multicolumn{1}{l|}{\cellcolor[HTML]{C0C0C0}***} & 46.69 & 5.67 & \multicolumn{1}{l|}{\cellcolor[HTML]{C0C0C0}***} & 53.13 & 5.68 & \multicolumn{1}{l|}{\cellcolor[HTML]{C0C0C0}***} & 53.46 & 3.99 & *** \\
\multicolumn{1}{l|}{ImP-Node:O.RTP} & 36.16 & 1.83 & \multicolumn{1}{l|}{***} & 22.24 & 1.25 & \multicolumn{1}{l|}{***} & 39.66 & 1.72 & \multicolumn{1}{l|}{***} & 20.16 & 1.12 & *** \\ \hline
 &  &  &  &  &  &  &  &  &  &  &  &  \\
 & \multicolumn{3}{c|}{lCS} & \multicolumn{3}{c|}{rUnc} & \multicolumn{3}{c|}{CCocc} & \multicolumn{3}{c}{CCsp} \\ \cline{2-13}
Smooth & \multicolumn{1}{c}{edf} & \multicolumn{1}{c}{F} & \multicolumn{1}{c|}{Sig} & \multicolumn{1}{c}{edf} & \multicolumn{1}{c}{F} & \multicolumn{1}{c|}{Sig} & \multicolumn{1}{c}{edf} & \multicolumn{1}{c}{F} & \multicolumn{1}{c|}{Sig} & \multicolumn{1}{c}{edf} & \multicolumn{1}{c}{F} & \multicolumn{1}{c}{Sig} \\ \hline
\multicolumn{1}{l|}{Node} & 5.77 & 4.06 & \multicolumn{1}{l|}{***} & 1.80 & 0.47 & \multicolumn{1}{l|}{} & 4.88 & 4.95 & \multicolumn{1}{l|}{***} & 3.90 & 3.45 & ** \\
\rowcolor[HTML]{C0C0C0}
\multicolumn{1}{l|}{\cellcolor[HTML]{C0C0C0}Days} & 1.00 & 3.68 & \multicolumn{1}{l|}{\cellcolor[HTML]{C0C0C0}} & 1.08 & 1.52 & \multicolumn{1}{l|}{\cellcolor[HTML]{C0C0C0}} & 1.19 & 0.72 & \multicolumn{1}{l|}{\cellcolor[HTML]{C0C0C0}} & 1.00 & 0.42 &  \\
\multicolumn{1}{l|}{Days-Node} & 37.10 & 2.04 & \multicolumn{1}{l|}{***} & 38.39 & 2.25 & \multicolumn{1}{l|}{***} & 35.40 & 1.60 & \multicolumn{1}{l|}{***} & 32.55 & 1.15 & ***
\end{tabular}

	\caption{Longitudinal tract interaction statistics. \textbf{Top}: Interaction of callosal tracts and ImPACT metrics. While significant non-flatness is detected for all ImPACT-Node interactions, note the reduction in effective degrees of freedom and F-stat between Post and RTP. VisMem = Visual Memory, TotSymp = Total Symptom, VisMot = Visual Motor. Node = global node, ImP:Base/Post/RTP = main effects of ImPACT metric for each group, ImP-Node = interaction term of node and ImPACT, ImP-Node:O.Post/RTP = Post/RTP group interaction as an ordered factor (relative to Base). \textbf{Bottom}: Interaction of select tracts and days between Post and RTP. edf = effective degrees of freedom, F = F-statistic, Sig = significance. *** = p$<$.001, ** = p$<$.01, * = p$<$.05. CCsf = callosal superior frontal, CCocc = callosal occipital, CCsp = callosal superior parietal, lCS = left corticospinal, rUnc = right uncinate.}
	\label{tbl:lgio-intx-cc}
\end{table}

Figure \ref{fig:lgio-intx-cc} visualizes Post and RTP tract node-FA-ImPACT difference interactions as topological surfaces to facilitate reviewing the pattern and magnitude of the tensor product interaction effects. A somewhat linear differential interaction at Post is found for each tract, and this difference from Base largely resolves by RTP. For instance, an interaction exists in the superior frontal tract about nodes 40-60 at Post (the same nodes which were implicated above; Section \ref{sssec:res-dwi-tract-tsa}) where lower visual memory scores are associated with decreased FA values (Figure \ref{fig:lgio-intx-cc}, A). At RTP, although a linear interaction is still somewhat present, the magnitude of the effect is diminished, suggesting both clinical and axonal recovery. Likewise, total symptoms are related to decreased FA values about superior frontal callosal nodes 40-60 (Figure \ref{fig:lgio-intx-cc}, B), poorer visual motor performance with nodes 40-50 of occipital (Figure \ref{fig:lgio-intx-cc}, C), and total symptoms with decreased FA superior parietal nodes 40-60 (Figure \ref{fig:lgio-intx-cc}, D). In each case of these cases, the FA-ImPACT interaction resolved by RTP, indicated by the `flatter' interaction difference surface (and fewer effective degrees of freedom and lower F-statistics in Table \ref{tbl:lgio-intx-cc}, top).

\begin{figure}[H]
	\centering
	\fbox{\includegraphics[width=0.95\textwidth]{fig_LGIO_intx_callosal.png}}
	\caption{HGAM tensor product interaction difference smooths, comparing Post and RTP node-FA-ImPACT interactions terms with Base. A linear interaction is present at Post in each quadrant (top plots), indicated by a gradient (Z-axis) change along the Y-axis at a specific node (X-axis) region. This interaction is diminished or not found at RTP (bottom plots).}
	\label{fig:lgio-intx-cc}
\end{figure}


% TODO: should select tract interaction results be kept?
\textit{Select Tract Interaction Results}. Further exploratory analyses were conducted to see whether scalar changes in selected tracts (Figure \ref{fig:lgio-gam-sel}) related to changes in ImPACT composite and total symptom scores. Surprisingly, of the six tracts and six ImPACT metrics tested, the hypothesized interaction was only detected between the left corticospinal tract and reaction time (Figure \ref{fig:lgio-intx-lcs}; Global $F_{(13.62, 13.94)}$ = 1695.79, \textit{p} $<$ .001; Impact:Base $F_{(1, 1)}$ = 0.2, \textit{p} $=$ .65; Impact:Post $F_{(1, 1)}$ = 0.05, \textit{p} $=$ .82; Impact:RTP $F_{(1, 1)}$ = 1.07, \textit{p} $=$ .29; Impact-Node $F_{(39.33, 76)}$ = 2.56, \textit{p} $<$ .001; Impact-Node:O.Post $F_{(44.17, 76)}$ = 2.29, \textit{p} $<$ .001; Impact-Node:O.RTP $F_{(44.36, 76)}$ = 1.64, \textit{p} $<$ .001). At Post, the slowest reaction times were associated with larger FA values about node 75. This pattern is reversed at RTP, likely due to sparsity in slow reaction times given recovery (see Figure \ref{fig:imp-gam}, bottom right).

\begin{figure}[H]
	\centering
	\fbox{\includegraphics[width=0.5\textwidth]{fig_LGIO_intx_lCS_rx_time.png}}
	\caption{HGAM tensor product interaction difference smooths, for the left corticospinal node-FA-reaction time interaction. An interaction about node 75 is present at Post that is not at RTP.}
	\label{fig:lgio-intx-lcs}
\end{figure}


\subsection{Tract FA Differences Relate to Recovery Time}
\label{ssec:res-dwi-time}
% TODO: check direction against tract smooths
We utilized tensor product interaction smooths to test whether RTP-Post $\Delta$FA values related to the number of days between Post and RTP as an exploratory analysis, as presumably more significant injuries (and their accompanying FA change) would relate to extended recovery times. From the tracts that demonstrated concussion injury- and recovery-related changes (Section \ref{ssec:res-dwi-tract}), four tracts demonstrated node-$\Delta$FA-time interactions that aligned with predictions: the left corticospinal, right uncinate, callosal superior parietal, and callosal orbital tracts (Table \ref{tbl:lgio-intx-cc}, bottom; Figure \ref{fig:di-time}), and within these tracts two interaction patterns were detected.

First, positive $\Delta$FA values were associated with longer recovery periods in the left corticospinal (node 50) and right uncinate (node 55) tracts (Figure \ref{fig:di-time}, A \& B), and negative $\Delta$FA values in the left corticospinal tracts related to a shorter recovery period. Second, negative RTP-Post $\Delta$FA values in the callosal superior parietal (node 55) and callosal orbital (node 60) tracts were related to longer recovery periods (Figure \ref{fig:di-time}, C \& D). As noted above, injury-related FA changes in this sample have largely been driven by changes in RD, and accordingly the former set of findings may indicate that demyelination deflated FA by increasing RD at Post, resulting in a positive RTP-Post $\Delta$FA value, while cellular edema may explain the increased Post FA values of the latter findings due to a decrease of RD (see Discussion). Shorter recovery periods, however, did not have a clear relationship with $\Delta$FA values in these tracts as indicated by the flat (homogeneous green) regions of the tensor product interaction smooths.

\begin{figure}[H]
	\centering
	\fbox{\includegraphics[width=0.95\textwidth]{fig_DI_time.png}}
	\caption{HGAM tensor product interaction smooths of FA change by recovery time. Interactions are present between RTP-Post $\Delta$FA values (Z-axis), number of days between Post and RTP visits (Y-axis), and tract nodes (X-axis). Positive RTP-Post $\Delta$FA values are indicated as yellow, and negative as blue, with gray bars indicating regions of extrapolation (insufficient data). Top plots indicate positive $\Delta$FA values about node 50 are associated with a larger number of days between Post and RTP, and bottom plots the opposite.}
	\label{fig:di-time}
\end{figure}


\section{Discussion}
\label{sec:disc}
% Recap
Diffusion MRI and ImPACT assessment data were collected from 69 University of Nebraska-Lincoln's NCAA athletes in order to study the effects of concussion on white matter organization. Athletes contributed data at three visits: pre-season (Base), within approximately 48 hours after diagnosed concussion (Post), and after completing the return-to-play protocol (RTP). This prospective, longitudinal dataset is singular, to our knowledge, and facilitates an important step forward in understanding the nature of injury and recovery. Probabilistic tractography generated a set of white matter tracts for each subject and visit, and we employed hierarchical generalized additive models (HGAM) to (a) conduct a longitudinal, whole-brain analysis of injury and recovery, (b) determine the source of FA change, and (c) relate changes in FA values to ImPACT and recovery time. Importantly, this is the first time HGAMs have been used to study concussion injury- and recovery-related tract scalar changes, and we propose that such a technique is critical given their capability of modeling non-linear interactions.

% Findings - scalar changes
\subsection{Scalar Changes Relate to Mechanisms of Injury}
\label{ssec:disc-tract}
In terms of study findings, we first detected within-tract FA changes in hypothesized regions (posterior corpus callosum) which recovered after injury, confirming our first hypothesis. Importantly, we also detected a number of FA changes in other tracts and demonstrated that, in accordance with our second hypothesis, RD was the source of such FA change. Identifying the source of FA change is critical to understanding injury sequelae and recovery as concussion-related changes in AD and RD are associated with different axonal pathologies.

For brief review (see \textcite{krieg2023IdentifyingPhenotypesDiffuse}), the axon is a thin, tubular extension of the neuronal cell body which is composed of the axolemma that anchors to the actin-spectrin complex and, when myelinated, are sheathed by concentric loops of glial lamellar extensions that attach to the cytoskeleton via anchoring proteins. Within the axon, $\alpha$/$\beta$-tubulins form microtubules which are connected with microtubule-associate proteins such as tau. These microtubules form the backbone of the axonal transport system, where transport proteins move mitochondria, vesicles, and other organelles along the axon \parencite{shin2020AxonalTransportDysfunction}. Neurofilaments occupy the space between the central microtubules and the actin-spectrin complex, and their sidearms support the diameter of the axon via electrostatic repulsion and by cross-bridging with neighboring filaments.

%  non/terminal apoptotic caspases can also be activated via increased oxidative stress and [Ca$^{2+}$]i.
With respect to injury, the rapid application of shear, tensile, and/or compressive strains from sports-related de/acceleration injury can have differential effects on the various components of the axon. Initially, excitotoxic activity and physical deformation can result in increased intracellular calcium concentrations ([Ca$^{2+}$]i) through depolarization of the membrane potential, axolemmal mechanoporation, and alterations of ionic channels \parencite{baracaldo-santamaria2022RevisitingExcitotoxicityTraumatic,christman1994UltrastructuralStudiesDiffuse,povlishock1997ImpactAccelerationInjury,pettus1996CharacterizationDistinctSet}. The increased [Ca$^{2+}$]i (via mechanoporation, voltage-gated Ca$^{2+}$ and Na$^+$ channels, AMPA and NMDA receptors, and reversal of Na$^+$-Ca$^{2+}$ exchangers) then triggers secondary injury mechanisms by activating proteolytic calpains which degrade cytoskeletal structures, ionic channels, and receptors \parencite{krieg2023IdentifyingPhenotypesDiffuse}. For instance, activated calpains hydrolyze nodal microtubule-association proteins, degrading the axonal transport system which results in nodal accumulation of transport products, causing swelling \parencite[i.e. varicosities, when stained for $\beta$-amyloid precursor protein][]{ma2013RoleCalpainsInjuryinduced,johnson2013AxonalPathologyTraumatic,shin2020AxonalTransportDysfunction}. Swelling increases tortuosity for intracellular water. Relatedly, increased intracellular osmolarity via excessive glutamatergic activity and disrupted/degraded ionic channels and exchangers draws water into the cell, resulting in cytotoxic edema \parencite{rungta2015CellularMechanismsNeuronal}. Extracellular water, having low tortuosity and relatively unconstrained diffusion, is associated with higher $\lambda_\perp$ values in diffusion MRI metrics while intra-axonal diffusion is both more tortuous and constrained \parencite{mayer2010ProspectiveDiffusionTensor,rosenblum2007CytotoxicEdemaMonitoring}. Nodal swelling into the extracellular space due to cellular or cytotoxic edema accordingly results in lower RD values due to the decrease of extracellular waters.

Further, secondary Ca$^{2+}$-mediated cascades can result in neurofilament compaction. In an uninjured state, phosphorylation of heavy and medium neurofilaments results in their straightening and bundling, supporting the axonal diameter. Injury-related increased [Ca$^{2+}$]i activates both calcineurin and calpain, which dephosphorylate and degrade the filaments, respectively, promoting compaction \parencite{krieg2023IdentifyingPhenotypesDiffuse,pant1988DephosphorylationNeurofilamentProteins,chen1999EvolutionNeurofilamentSubtype}. This compaction increases tortuosity along the axon, decreasing $\lambda_\parallel$, as would axonal undulation and caspase-mediated axotomy \parencite{svandova2023MakingHeadCaspases}. Finally, myelin binding sites near nodes of Ranvier are vulnerable, as are nodal channels, with injury resulting in channel loss, demyelination, and the detection of disassociated key anchoring and structural proteins \parencite{song2022ConcussionLeadsWidespread,zhu2016NodalTotalAxonal,krieg2023IdentifyingPhenotypesDiffuse}. As demyelination occurs, encapsulated water within the sheathes is released into the extracellular space, increasing $\lambda_\perp$ \parencite{mayer2010ProspectiveDiffusionTensor}.

Together, then, $\lambda_\parallel$/AD can be understood to relate to key structural aspects of the axon, with a decrease indicative of severe damage from which the cell may not be able to recover. On the other hand, changes in $\lambda_\perp$/RD relate to intra/extracellular water ratios and their respective differences in tortuosity, which can be driven by cellular or cytotoxic edema and/or demyelination \parencite{borja2018DiffusionMRImaging,mayer2017SpectrumMildTraumatic,lindsey2023DiffusionWeightedImagingMild}. Understanding both is critical when investigating injury-related FA changes as FA is calculated from $\lambda_{1-3}$, where a decrease in FA can result from a lower AD \textit{or} higher RD value. Our first set of findings indicate that the concussion-related FA differences at Post were largely driven by changes to RD and not AD, which is consistent with both the severity of the injury (sport-related concussion or mTBI) and axonal damage from which the cell is able to recover; cellular repair is supported by RTP FA and RD values which returned to baseline. Further, the majority of detected patterns of scalar change (increased FA, decreased RD) are consistent with cellular edema. For tracts which worsened (callosum orbital, left arcuate), the relative pattern of changes were similar: both tracts demonstrated decreased FA and increased RD at RTP (relative to base), a pattern that is consistent with demyelination.


\subsection{FA Relates to ImPACT and Recovery Time}
\label{ssec:disc-impact}
% Findings - scalar and impact interactions
Second, we detected our hypothesized interaction between changes in ImPACT assessment and tract FA values in a number of tracts: the callosum superior frontal FA values related to both the Visual Memory composite and Total Symptom score, callosum occipital to Visual Memory composite, callosum superior parietal to Total Symptom score, and left corticospinal to Reaction Time composite. We note that our hypothesized interaction pattern was only detected in these five instances; while it is possible that a non-linear relationship between node FA values and ImPACT may better capture the nature of injury, or that changes in another scalar would share more variance with ImPACT measures, these were not originally hypothesized, and we found them difficult to understand and justify. We suspect that the lack of hypothesized interactions speak to the limited specificity of ImPACT metrics \parencite{schatz2013SensitivitySpecificityOnline} when studying concussion injury- and recovery-related diffusion changes.

Nevertheless, these multimodal tensor product interaction smooths directly relate clinical diagnostic metrics to within-tract diffusion MRI values, and also help to tease apart injury heterogeneity. Indeed, one difficulty in studying sport-related concussion is the variable nature and location of injury, which constitutes significant inter-individual differences at the group level \parencite{lindsey2023DiffusionWeightedImagingMild}. Accordingly, group-level statistics likely suffer from Type-II errors, where a real but idiosyncratic injury for one athlete is lost in the total variance of the group. The ability to incorporate clinical metrics by modeling data with 2D tensor product interaction smooths help interrogate such group data and may reveal subgroups. For instance, in Figure \ref{fig:lgio-intx-cc}D (top) only individuals with large decreases in FA about node 50 had higher Total Symptom scores, while the tract-FA-ImPACT relationship was rather flat for athletes without such an injury. Additionally, we note that such an interaction is only apparent at the Post visit, during acute injury and poor ImPACT performance (Figure \ref{fig:imp-gam}). By RTP, when ImPACT performance recovers and athletes are cleared for play, the tract-FA-ImPACT relationships do not differ from baseline (Figure \ref{fig:lgio-intx-cc}D, bottom). This pattern of findings, where the hypothesized interaction between tract node FA and ImPACT metric is present at Post and not RTP, extends to all those reported. Given that evidence of recovery from Figures \ref{fig:lgio-gam-cc} \& \ref{fig:lgio-gam-sel}, we interpret such patterns as, first, evidence that microstructural disruption of axonal processes relate to a decline in ImPACT performance, and second that the lack of such tract-FA-ImPACT interaction at RTP is evidence of recovery.

% Findings - scalar change and recovery time
Third, and finally, we tested whether recovery time was related to the difference in Post and RTP FA values. We reasoned that given all athletes were eventually cleared for return-to-play and that tract scalars were demonstrated to return to Base values, a larger RTP-Post $\Delta$FA would be driven by more severe injury sequelae at Post and that such injuries would necessitate longer recovery periods. The tensor product interaction smooths of Figure \ref{fig:di-time} demonstrate a linear relationship between FA change and recovery time in the left corticospinal, right uncinate, callosum superior parietal, and callosum orbital tracts. Within these tracts, two patterns were detected: positive $\Delta$FA values related to longer recovery periods in the corticospinal and uncinate tracts (Figure \ref{fig:di-time}, top), presumably driven by RD increase due to demyelination at Post, and the reverse in callosal superior parietal and orbital tracts (Figure \ref{fig:di-time}, bottom) which may be caused by RD reduction due to cellular or cytotoxic edema. Such findings provide insight into variance in recovery trajectories, and we note that tensor product interaction smooths would be well employed to longitudinally study recovery if multiple scans were acquired during the recovery period. Additionally, it would be possible to include other factors which may relate to recovery (e.g. metrics of cardiovascular exercise or number of previous TBIs) to determine the amount of shared variance in recovery.


\subsection{Limitations}
\label{ssec:disc-limit}
While HGAMs provide a powerful statistical approach to model within-tract scalar values across groups or time, a number of limitations exist in the approach reported above. First, we elected to start with a longitudinal, whole-brain model which incorporated all tracts from each participant across all visits in order to capture any shared variance of injury which spanned multiple tracts. In specifying such a model, each tract was provided with the same maximum basis dimension, a parameter which controls the `wiggliness' of the smooth. As tractometric profiles differ in their curvature, such a specification may have resulted in underfitting certain tracts. While a review of each tract, and models specific for each tract, indicated that a basis dimension of 15 was sufficient, future models may require fine-tuning this parameter. Second, interpreting smooths for significance versus evidence about the intended hypothesis can be an involved process. Here, we used HGAMs to investigate concussion-related injury and recovery, and selected models around a central hypothesis that Post tractometric profiles and their corresponding smooths, but not those for RTP, would differ from Base. GAMs test against the null hypothesis, however, that the best fit of the interaction is a flat smooth, or, that increasing the effective degrees of freedom (wiggliness) does not improve model fit. Accordingly, simply modeling a tract profile would result in significant models given the curvature of the node-scalar interaction. To account for this, we both employed HGAMs which enabled us to hold constant the tract curvature (global smooth) and ordinal factors (Base $<$ Post $<$ RTP) to derive test statistics that related to our hypotheses. Third, in addition to the `significance' of the statistic, it is critical to consider the magnitude of the effect, and an a priori idea of the shape of the smooth is warranted when investigating complex interactions. While smooths are penalized to reduce overfitting, small deflections from flatness can result in statistical, but not practical, significance. To guard against this, we interrogated the magnitude of the partial effects for the smooths of interest, often with respect to the global smooth, and started with defined hypotheses of the deformation pattern in tensor product interaction smooths. Nevertheless, we considered certain effects as not (practically) significant, such as the RTP vs Base interaction smooths, while their corresponding p-values were less than the criterion $\alpha$ = .05, and note that in such cases both the effective degrees of freedom and F-statistics were reduced. Fourth, the statistic of tensor product interaction smooths can be driven by sparsity: when investigating the interaction of RTP vs Post $\Delta$FA and recovery time, relatively few athletes had recovery times longer than 14 days and may have driven the deflections from flatness (particularly in Figure \ref{fig:di-time} C \& D). We nevertheless included these results as we considered such models to be instructive. Fifth, sport related concussion is a category of injury that is defined by its heterogeneity and accordingly group-level statistics lack sensitivity. We initially sought to identify subgroups within our sample, which would allow for a double-dissociation approach to studying tractometric changes, but our sample was too uniform in their ImPACT composites to identify clusters. Larger datasets would likely have a number of subgroups which would allow for the better interrogation of the relationship between structural changes and clinical response. Finally, while we quantified the contribution of algorithmic (run-rerun) variance, it is relevant to consider scan-rescan variance when studying mTBI with diffusion weighted imaging. Modern, faster protocols that allow for multiple scan acquisitions during a single visit (e.g. \cite{li2020EvaluationMultishellDiffusion}) will account for some inherent thermal noise as multiple measures of each diffusion direction can be combined and averaged.


\subsection{Conclusions}
\label{ssec:disc-conc}
The present study demonstrates that HGAMs are well-met to sensitively model within-tract diffusion changes that relate to concussion injury and recovery. Their ability to model non-linear interactions provides a better fit to complex data than traditional approaches, facilitating the detection of subtle group-related changes. Further, longitudinal or group-difference models are facilitated by incorporating a nested factor structure, and higher dimensional multimodal models allow for the interrogation of clinical import. Here, we employed HGAMs to demonstrate that concussion injury- and recovery-related scalar changes were present in a number of commissural and association tracts, that radial and not axial diffusivity was the source of fractional anisotropy change, and that such changes related to both clinical metrics and recovery duration.


% acknowledgment page
\section*{Acknowledgments}
\label{sec:ack}
The authors thank Ariana M. Hedges-Muncy for statistical consults and Joanne Murray for MRI data acquisition. This work was supported by a grant awarded to MN by the National Institute of Health (NIH) Great Plains IDeA-CTR Network, Pilot Grant Program (2018-2020) "Brain connectivity for prediction of lesion site in sports-related concussion".


% write bibliography
\pagebreak
\printbibliography
\pagebreak


% make supplemental
\section{Supplemental Materials}
\label{sec:supp-materials}
\beginsupplement


\begin{equ}[H]
	\begin{lstlisting}
		fit_LGI <- mgcv::bam(
		  <scalar> ~ s(subj_id, scan_name, bs="re") +
		    s(node_id, bs="tp", k=15, m=2) +
		    s(node_id, by=scan_name, bs="tp", k=15, m=1),
		  data=df,
		  family=<family>,
		  method="fREML",
		  nthreads=4
		)
	\end{lstlisting}
	\caption{Tract scalars are modeled as a function of tract node with thin-plate regression splines using both global and group (\lstinline{scan_name}) smooths as well as individual group wiggliness. \lstinline{<scalar>} = relevant DWI metric (AD, RD, MD, or FA), \lstinline{scan_name} = visit identifier factor (Base, Post, RTP), \lstinline{<family>} = relevant family and link function for scalar distribution.}
	\label{supp-code:gam-lgi}
\end{equ}


\begin{equ}[H]
	\begin{lstlisting}
		fit_LGI_intx <- mgcv::bam(
		  <scalar> ~ s(subj_id, scan_name, bs="re") +
		    s(node_id, bs="tp", k=15, m=2) +
		    s(imp_meas, by=scan_name, bs="tp", k=5) +
		    ti(
		      node_id, imp_meas, by=scan_name,
		      bs=c("tp","tp"), k=c(20,5), m=1
		    ),
		  data=df,
		  family=<family>,
		  method="fREML",
		  nthreads=4
		)
	\end{lstlisting}
	\caption{Tract scalars are modeled as a function of separate 1D node and ImPACT smooths as well as a 2D tensor product interaction surface. \lstinline{imp_meas} = ImPACT composite or total symptom measure.}
	\label{supp-code:gam-lgi-intx}
\end{equ}


\begin{equ}[H]
	\begin{lstlisting}
		fit_G <- mgcv::bam(
		  imp_meas ~ s(subj_id, bs="re") +
		    s(num_assess, bs="tp", k=3),
		  data=df,
		  family=<family>,
		  method="fREML"
		)
	\end{lstlisting}
	\caption{ImPACT metrics modeled as a function of number of assessments using a single global smooth. \lstinline{imp_meas} = ImPACT composite or total symptom score, \lstinline{num_assess} = assessment number (1=Base, 2=Post, 3=RTP).}
	\label{supp-code:gam-impact}
\end{equ}


\begin{equ}[H]
	\begin{lstlisting}
		fit_DI <- mgcv::bam(
		  delta_fa ~ s(subj_id, by=tract_name, bs="re") +
		    s(node_id, by=tract_name, bs="tp", k=15) +
		    tract_name,
		  data=df,
		  family=gaussian(),
		  method="fREML",
		  nthreads=12
		)
	\end{lstlisting}
	\caption{Run-Rerun $\Delta$FA values were modeled with node smooths for each tract.}
	\label{supp-code:gam-di}
\end{equ}

% \subsection{Tables}
% \label{ssec:supp-tables}
% Supplemental Tables.


% \subsection{Figures}
% \label{ssec:supp-figures}
% Supplemental Figures.


\end{document}